%
% API Documentation for API Documentation
% Module Epigrass.testinst
%
% Generated by epydoc 3.0beta1
% [Tue May 29 09:37:54 2007]
%

%%%%%%%%%%%%%%%%%%%%%%%%%%%%%%%%%%%%%%%%%%%%%%%%%%%%%%%%%%%%%%%%%%%%%%%%%%%
%%                          Module Description                           %%
%%%%%%%%%%%%%%%%%%%%%%%%%%%%%%%%%%%%%%%%%%%%%%%%%%%%%%%%%%%%%%%%%%%%%%%%%%%

    \index{Epigrass \textit{(package)}!Epigrass.testinst \textit{(module)}|(}
\section{Module Epigrass.testinst}

    \label{Epigrass:testinst}
Unit testing script


%%%%%%%%%%%%%%%%%%%%%%%%%%%%%%%%%%%%%%%%%%%%%%%%%%%%%%%%%%%%%%%%%%%%%%%%%%%
%%                           Class Description                           %%
%%%%%%%%%%%%%%%%%%%%%%%%%%%%%%%%%%%%%%%%%%%%%%%%%%%%%%%%%%%%%%%%%%%%%%%%%%%

    \index{Epigrass \textit{(package)}!Epigrass.testinst \textit{(module)}!Epigrass.testinst.testObjInstantiation \textit{(class)}|(}
\subsection{Class testObjInstantiation}

    \label{Epigrass:testinst:testObjInstantiation}
\begin{tabular}{cccccc}
% Line for unittest.TestCase, linespec=[False]
\multicolumn{2}{r}{\settowidth{\BCL}{unittest.TestCase}\multirow{2}{\BCL}{unittest.TestCase}}
&&
  \\\cline{3-3}
  &&\multicolumn{1}{c|}{}
&&
  \\
&&\multicolumn{2}{l}{\textbf{Epigrass.testinst.testObjInstantiation}}
\end{tabular}


%%%%%%%%%%%%%%%%%%%%%%%%%%%%%%%%%%%%%%%%%%%%%%%%%%%%%%%%%%%%%%%%%%%%%%%%%%%
%%                                Methods                                %%
%%%%%%%%%%%%%%%%%%%%%%%%%%%%%%%%%%%%%%%%%%%%%%%%%%%%%%%%%%%%%%%%%%%%%%%%%%%

  \subsubsection{Methods}

    \vspace{0.5ex}

    \begin{boxedminipage}{\textwidth}

    \raggedright \textbf{setUp}(\textit{self})

    Hook method for setting up the test fixture before exercising it.

    \vspace{1ex}

      Overrides: unittest.TestCase.setUp 	extit{(inherited documentation)}

    \end{boxedminipage}

    \label{Epigrass:testinst:testObjInstantiation:testSites}
    \index{Epigrass \textit{(package)}!Epigrass.testinst \textit{(module)}!Epigrass.testinst.testObjInstantiation \textit{(class)}!Epigrass.testinst.testObjInstantiation.testSites \textit{(method)}}

    \vspace{0.5ex}

    \begin{boxedminipage}{\textwidth}

    \raggedright \textbf{testSites}(\textit{self})

    \end{boxedminipage}

    \label{Epigrass:testinst:testObjInstantiation:testEdges}
    \index{Epigrass \textit{(package)}!Epigrass.testinst \textit{(module)}!Epigrass.testinst.testObjInstantiation \textit{(class)}!Epigrass.testinst.testObjInstantiation.testEdges \textit{(method)}}

    \vspace{0.5ex}

    \begin{boxedminipage}{\textwidth}

    \raggedright \textbf{testEdges}(\textit{self})

    \end{boxedminipage}

    \label{Epigrass:testinst:testObjInstantiation:testGraph}
    \index{Epigrass \textit{(package)}!Epigrass.testinst \textit{(module)}!Epigrass.testinst.testObjInstantiation \textit{(class)}!Epigrass.testinst.testObjInstantiation.testGraph \textit{(method)}}

    \vspace{0.5ex}

    \begin{boxedminipage}{\textwidth}

    \raggedright \textbf{testGraph}(\textit{self})

    \end{boxedminipage}

    \label{unittest:TestCase:__call__}
    \index{unittest.TestCase.\_\_call\_\_ \textit{(function)}}

    \vspace{0.5ex}

    \begin{boxedminipage}{\textwidth}

    \raggedright \textbf{\_\_call\_\_}(\textit{self}, *\textit{args}, **\textit{kwds})

    \end{boxedminipage}

    \label{unittest:TestCase:__init__}
    \index{unittest.TestCase.\_\_init\_\_ \textit{(function)}}

    \vspace{0.5ex}

    \begin{boxedminipage}{\textwidth}

    \raggedright \textbf{\_\_init\_\_}(\textit{self}, \textit{methodName}=\texttt{'runTest'})

    \vspace{-1.5ex}

    \rule{\textwidth}{0.5\fboxrule}
    Create an instance of the class that will use the named test method 
    when executed. Raises a ValueError if the instance does not have a 
    method with the specified name.

    \vspace{1ex}

    \end{boxedminipage}

    \label{unittest:TestCase:__repr__}
    \index{unittest.TestCase.\_\_repr\_\_ \textit{(function)}}

    \vspace{0.5ex}

    \begin{boxedminipage}{\textwidth}

    \raggedright \textbf{\_\_repr\_\_}(\textit{self})

    \end{boxedminipage}

    \label{unittest:TestCase:__str__}
    \index{unittest.TestCase.\_\_str\_\_ \textit{(function)}}

    \vspace{0.5ex}

    \begin{boxedminipage}{\textwidth}

    \raggedright \textbf{\_\_str\_\_}(\textit{self})

    \end{boxedminipage}

    \label{unittest:TestCase:failUnlessAlmostEqual}
    \index{unittest.TestCase.failUnlessAlmostEqual \textit{(function)}}

    \vspace{0.5ex}

    \begin{boxedminipage}{\textwidth}

    \raggedright \textbf{assertAlmostEqual}(\textit{self}, \textit{first}, \textit{second}, \textit{places}=\texttt{7}, \textit{msg}=\texttt{None})

    \vspace{-1.5ex}

    \rule{\textwidth}{0.5\fboxrule}
    Fail if the two objects are unequal as determined by their difference 
    rounded to the given number of decimal places (default 7) and comparing
    to zero.

    Note that decimal places (from zero) are usually not the same as 
    significant digits (measured from the most signficant digit).

    \vspace{1ex}

    \end{boxedminipage}

    \label{unittest:TestCase:failUnlessAlmostEqual}
    \index{unittest.TestCase.failUnlessAlmostEqual \textit{(function)}}

    \vspace{0.5ex}

    \begin{boxedminipage}{\textwidth}

    \raggedright \textbf{assertAlmostEquals}(\textit{self}, \textit{first}, \textit{second}, \textit{places}=\texttt{7}, \textit{msg}=\texttt{None})

    \vspace{-1.5ex}

    \rule{\textwidth}{0.5\fboxrule}
    Fail if the two objects are unequal as determined by their difference 
    rounded to the given number of decimal places (default 7) and comparing
    to zero.

    Note that decimal places (from zero) are usually not the same as 
    significant digits (measured from the most signficant digit).

    \vspace{1ex}

    \end{boxedminipage}

    \label{unittest:TestCase:failUnlessEqual}
    \index{unittest.TestCase.failUnlessEqual \textit{(function)}}

    \vspace{0.5ex}

    \begin{boxedminipage}{\textwidth}

    \raggedright \textbf{assertEqual}(\textit{self}, \textit{first}, \textit{second}, \textit{msg}=\texttt{None})

    \vspace{-1.5ex}

    \rule{\textwidth}{0.5\fboxrule}
    Fail if the two objects are unequal as determined by the '==' operator.

    \vspace{1ex}

    \end{boxedminipage}

    \label{unittest:TestCase:failUnlessEqual}
    \index{unittest.TestCase.failUnlessEqual \textit{(function)}}

    \vspace{0.5ex}

    \begin{boxedminipage}{\textwidth}

    \raggedright \textbf{assertEquals}(\textit{self}, \textit{first}, \textit{second}, \textit{msg}=\texttt{None})

    \vspace{-1.5ex}

    \rule{\textwidth}{0.5\fboxrule}
    Fail if the two objects are unequal as determined by the '==' operator.

    \vspace{1ex}

    \end{boxedminipage}

    \label{unittest:TestCase:failIf}
    \index{unittest.TestCase.failIf \textit{(function)}}

    \vspace{0.5ex}

    \begin{boxedminipage}{\textwidth}

    \raggedright \textbf{assertFalse}(\textit{self}, \textit{expr}, \textit{msg}=\texttt{None})

    \vspace{-1.5ex}

    \rule{\textwidth}{0.5\fboxrule}
    Fail the test if the expression is true.

    \vspace{1ex}

    \end{boxedminipage}

    \label{unittest:TestCase:failIfAlmostEqual}
    \index{unittest.TestCase.failIfAlmostEqual \textit{(function)}}

    \vspace{0.5ex}

    \begin{boxedminipage}{\textwidth}

    \raggedright \textbf{assertNotAlmostEqual}(\textit{self}, \textit{first}, \textit{second}, \textit{places}=\texttt{7}, \textit{msg}=\texttt{None})

    \vspace{-1.5ex}

    \rule{\textwidth}{0.5\fboxrule}
    Fail if the two objects are equal as determined by their difference 
    rounded to the given number of decimal places (default 7) and comparing
    to zero.

    Note that decimal places (from zero) are usually not the same as 
    significant digits (measured from the most signficant digit).

    \vspace{1ex}

    \end{boxedminipage}

    \label{unittest:TestCase:failIfAlmostEqual}
    \index{unittest.TestCase.failIfAlmostEqual \textit{(function)}}

    \vspace{0.5ex}

    \begin{boxedminipage}{\textwidth}

    \raggedright \textbf{assertNotAlmostEquals}(\textit{self}, \textit{first}, \textit{second}, \textit{places}=\texttt{7}, \textit{msg}=\texttt{None})

    \vspace{-1.5ex}

    \rule{\textwidth}{0.5\fboxrule}
    Fail if the two objects are equal as determined by their difference 
    rounded to the given number of decimal places (default 7) and comparing
    to zero.

    Note that decimal places (from zero) are usually not the same as 
    significant digits (measured from the most signficant digit).

    \vspace{1ex}

    \end{boxedminipage}

    \label{unittest:TestCase:failIfEqual}
    \index{unittest.TestCase.failIfEqual \textit{(function)}}

    \vspace{0.5ex}

    \begin{boxedminipage}{\textwidth}

    \raggedright \textbf{assertNotEqual}(\textit{self}, \textit{first}, \textit{second}, \textit{msg}=\texttt{None})

    \vspace{-1.5ex}

    \rule{\textwidth}{0.5\fboxrule}
    Fail if the two objects are equal as determined by the '==' operator.

    \vspace{1ex}

    \end{boxedminipage}

    \label{unittest:TestCase:failIfEqual}
    \index{unittest.TestCase.failIfEqual \textit{(function)}}

    \vspace{0.5ex}

    \begin{boxedminipage}{\textwidth}

    \raggedright \textbf{assertNotEquals}(\textit{self}, \textit{first}, \textit{second}, \textit{msg}=\texttt{None})

    \vspace{-1.5ex}

    \rule{\textwidth}{0.5\fboxrule}
    Fail if the two objects are equal as determined by the '==' operator.

    \vspace{1ex}

    \end{boxedminipage}

    \label{unittest:TestCase:failUnlessRaises}
    \index{unittest.TestCase.failUnlessRaises \textit{(function)}}

    \vspace{0.5ex}

    \begin{boxedminipage}{\textwidth}

    \raggedright \textbf{assertRaises}(\textit{self}, \textit{excClass}, \textit{callableObj}, *\textit{args}, **\textit{kwargs})

    \vspace{-1.5ex}

    \rule{\textwidth}{0.5\fboxrule}
    Fail unless an exception of class excClass is thrown by callableObj 
    when invoked with arguments args and keyword arguments kwargs. If a 
    different type of exception is thrown, it will not be caught, and the 
    test case will be deemed to have suffered an error, exactly as for an 
    unexpected exception.

    \vspace{1ex}

    \end{boxedminipage}

    \label{unittest:TestCase:failUnless}
    \index{unittest.TestCase.failUnless \textit{(function)}}

    \vspace{0.5ex}

    \begin{boxedminipage}{\textwidth}

    \raggedright \textbf{assertTrue}(\textit{self}, \textit{expr}, \textit{msg}=\texttt{None})

    \vspace{-1.5ex}

    \rule{\textwidth}{0.5\fboxrule}
    Fail the test unless the expression is true.

    \vspace{1ex}

    \end{boxedminipage}

    \label{unittest:TestCase:failUnless}
    \index{unittest.TestCase.failUnless \textit{(function)}}

    \vspace{0.5ex}

    \begin{boxedminipage}{\textwidth}

    \raggedright \textbf{assert\_}(\textit{self}, \textit{expr}, \textit{msg}=\texttt{None})

    \vspace{-1.5ex}

    \rule{\textwidth}{0.5\fboxrule}
    Fail the test unless the expression is true.

    \vspace{1ex}

    \end{boxedminipage}

    \label{unittest:TestCase:countTestCases}
    \index{unittest.TestCase.countTestCases \textit{(function)}}

    \vspace{0.5ex}

    \begin{boxedminipage}{\textwidth}

    \raggedright \textbf{countTestCases}(\textit{self})

    \end{boxedminipage}

    \label{unittest:TestCase:debug}
    \index{unittest.TestCase.debug \textit{(function)}}

    \vspace{0.5ex}

    \begin{boxedminipage}{\textwidth}

    \raggedright \textbf{debug}(\textit{self})

    \vspace{-1.5ex}

    \rule{\textwidth}{0.5\fboxrule}
    Run the test without collecting errors in a TestResult

    \vspace{1ex}

    \end{boxedminipage}

    \label{unittest:TestCase:defaultTestResult}
    \index{unittest.TestCase.defaultTestResult \textit{(function)}}

    \vspace{0.5ex}

    \begin{boxedminipage}{\textwidth}

    \raggedright \textbf{defaultTestResult}(\textit{self})

    \end{boxedminipage}

    \label{unittest:TestCase:fail}
    \index{unittest.TestCase.fail \textit{(function)}}

    \vspace{0.5ex}

    \begin{boxedminipage}{\textwidth}

    \raggedright \textbf{fail}(\textit{self}, \textit{msg}=\texttt{None})

    \vspace{-1.5ex}

    \rule{\textwidth}{0.5\fboxrule}
    Fail immediately, with the given message.

    \vspace{1ex}

    \end{boxedminipage}

    \label{unittest:TestCase:failIf}
    \index{unittest.TestCase.failIf \textit{(function)}}

    \vspace{0.5ex}

    \begin{boxedminipage}{\textwidth}

    \raggedright \textbf{failIf}(\textit{self}, \textit{expr}, \textit{msg}=\texttt{None})

    \vspace{-1.5ex}

    \rule{\textwidth}{0.5\fboxrule}
    Fail the test if the expression is true.

    \vspace{1ex}

    \end{boxedminipage}

    \label{unittest:TestCase:failIfAlmostEqual}
    \index{unittest.TestCase.failIfAlmostEqual \textit{(function)}}

    \vspace{0.5ex}

    \begin{boxedminipage}{\textwidth}

    \raggedright \textbf{failIfAlmostEqual}(\textit{self}, \textit{first}, \textit{second}, \textit{places}=\texttt{7}, \textit{msg}=\texttt{None})

    \vspace{-1.5ex}

    \rule{\textwidth}{0.5\fboxrule}
    Fail if the two objects are equal as determined by their difference 
    rounded to the given number of decimal places (default 7) and comparing
    to zero.

    Note that decimal places (from zero) are usually not the same as 
    significant digits (measured from the most signficant digit).

    \vspace{1ex}

    \end{boxedminipage}

    \label{unittest:TestCase:failIfEqual}
    \index{unittest.TestCase.failIfEqual \textit{(function)}}

    \vspace{0.5ex}

    \begin{boxedminipage}{\textwidth}

    \raggedright \textbf{failIfEqual}(\textit{self}, \textit{first}, \textit{second}, \textit{msg}=\texttt{None})

    \vspace{-1.5ex}

    \rule{\textwidth}{0.5\fboxrule}
    Fail if the two objects are equal as determined by the '==' operator.

    \vspace{1ex}

    \end{boxedminipage}

    \label{unittest:TestCase:failUnless}
    \index{unittest.TestCase.failUnless \textit{(function)}}

    \vspace{0.5ex}

    \begin{boxedminipage}{\textwidth}

    \raggedright \textbf{failUnless}(\textit{self}, \textit{expr}, \textit{msg}=\texttt{None})

    \vspace{-1.5ex}

    \rule{\textwidth}{0.5\fboxrule}
    Fail the test unless the expression is true.

    \vspace{1ex}

    \end{boxedminipage}

    \label{unittest:TestCase:failUnlessAlmostEqual}
    \index{unittest.TestCase.failUnlessAlmostEqual \textit{(function)}}

    \vspace{0.5ex}

    \begin{boxedminipage}{\textwidth}

    \raggedright \textbf{failUnlessAlmostEqual}(\textit{self}, \textit{first}, \textit{second}, \textit{places}=\texttt{7}, \textit{msg}=\texttt{None})

    \vspace{-1.5ex}

    \rule{\textwidth}{0.5\fboxrule}
    Fail if the two objects are unequal as determined by their difference 
    rounded to the given number of decimal places (default 7) and comparing
    to zero.

    Note that decimal places (from zero) are usually not the same as 
    significant digits (measured from the most signficant digit).

    \vspace{1ex}

    \end{boxedminipage}

    \label{unittest:TestCase:failUnlessEqual}
    \index{unittest.TestCase.failUnlessEqual \textit{(function)}}

    \vspace{0.5ex}

    \begin{boxedminipage}{\textwidth}

    \raggedright \textbf{failUnlessEqual}(\textit{self}, \textit{first}, \textit{second}, \textit{msg}=\texttt{None})

    \vspace{-1.5ex}

    \rule{\textwidth}{0.5\fboxrule}
    Fail if the two objects are unequal as determined by the '==' operator.

    \vspace{1ex}

    \end{boxedminipage}

    \label{unittest:TestCase:failUnlessRaises}
    \index{unittest.TestCase.failUnlessRaises \textit{(function)}}

    \vspace{0.5ex}

    \begin{boxedminipage}{\textwidth}

    \raggedright \textbf{failUnlessRaises}(\textit{self}, \textit{excClass}, \textit{callableObj}, *\textit{args}, **\textit{kwargs})

    \vspace{-1.5ex}

    \rule{\textwidth}{0.5\fboxrule}
    Fail unless an exception of class excClass is thrown by callableObj 
    when invoked with arguments args and keyword arguments kwargs. If a 
    different type of exception is thrown, it will not be caught, and the 
    test case will be deemed to have suffered an error, exactly as for an 
    unexpected exception.

    \vspace{1ex}

    \end{boxedminipage}

    \label{unittest:TestCase:id}
    \index{unittest.TestCase.id \textit{(function)}}

    \vspace{0.5ex}

    \begin{boxedminipage}{\textwidth}

    \raggedright \textbf{id}(\textit{self})

    \end{boxedminipage}

    \label{unittest:TestCase:run}
    \index{unittest.TestCase.run \textit{(function)}}

    \vspace{0.5ex}

    \begin{boxedminipage}{\textwidth}

    \raggedright \textbf{run}(\textit{self}, \textit{result}=\texttt{None})

    \end{boxedminipage}

    \label{unittest:TestCase:shortDescription}
    \index{unittest.TestCase.shortDescription \textit{(function)}}

    \vspace{0.5ex}

    \begin{boxedminipage}{\textwidth}

    \raggedright \textbf{shortDescription}(\textit{self})

    \vspace{-1.5ex}

    \rule{\textwidth}{0.5\fboxrule}
    Returns a one-line description of the test, or None if no description 
    has been provided.

    The default implementation of this method returns the first line of the
    specified test method's docstring.

    \vspace{1ex}

    \end{boxedminipage}

    \label{unittest:TestCase:tearDown}
    \index{unittest.TestCase.tearDown \textit{(function)}}

    \vspace{0.5ex}

    \begin{boxedminipage}{\textwidth}

    \raggedright \textbf{tearDown}(\textit{self})

    \vspace{-1.5ex}

    \rule{\textwidth}{0.5\fboxrule}
    Hook method for deconstructing the test fixture after testing it.

    \vspace{1ex}

    \end{boxedminipage}

    \index{Epigrass \textit{(package)}!Epigrass.testinst \textit{(module)}!Epigrass.testinst.testObjInstantiation \textit{(class)}|)}
    \index{Epigrass \textit{(package)}!Epigrass.testinst \textit{(module)}|)}
