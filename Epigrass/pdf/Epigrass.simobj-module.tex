%
% API Documentation for API Documentation
% Module Epigrass.simobj
%
% Generated by epydoc 3.0beta1
% [Tue May 29 09:37:54 2007]
%

%%%%%%%%%%%%%%%%%%%%%%%%%%%%%%%%%%%%%%%%%%%%%%%%%%%%%%%%%%%%%%%%%%%%%%%%%%%
%%                          Module Description                           %%
%%%%%%%%%%%%%%%%%%%%%%%%%%%%%%%%%%%%%%%%%%%%%%%%%%%%%%%%%%%%%%%%%%%%%%%%%%%

    \index{Epigrass \textit{(package)}!Epigrass.simobj \textit{(module)}|(}
\section{Module Epigrass.simobj}

    \label{Epigrass:simobj}
This Module contains the definitions of objects for spatial simulation on 
geo reference spaces.


%%%%%%%%%%%%%%%%%%%%%%%%%%%%%%%%%%%%%%%%%%%%%%%%%%%%%%%%%%%%%%%%%%%%%%%%%%%
%%                           Class Description                           %%
%%%%%%%%%%%%%%%%%%%%%%%%%%%%%%%%%%%%%%%%%%%%%%%%%%%%%%%%%%%%%%%%%%%%%%%%%%%

    \index{Epigrass \textit{(package)}!Epigrass.simobj \textit{(module)}!Epigrass.simobj.siteobj \textit{(class)}|(}
\subsection{Class siteobj}

    \label{Epigrass:simobj:siteobj}
Basic site object containing attributes and methods common to all site 
objects.


%%%%%%%%%%%%%%%%%%%%%%%%%%%%%%%%%%%%%%%%%%%%%%%%%%%%%%%%%%%%%%%%%%%%%%%%%%%
%%                                Methods                                %%
%%%%%%%%%%%%%%%%%%%%%%%%%%%%%%%%%%%%%%%%%%%%%%%%%%%%%%%%%%%%%%%%%%%%%%%%%%%

  \subsubsection{Methods}

    \label{Epigrass:simobj:siteobj:__init__}
    \index{Epigrass \textit{(package)}!Epigrass.simobj \textit{(module)}!Epigrass.simobj.siteobj \textit{(class)}!Epigrass.simobj.siteobj.\_\_init\_\_ \textit{(method)}}

    \vspace{0.5ex}

    \begin{boxedminipage}{\textwidth}

    \raggedright \textbf{\_\_init\_\_}(\textit{self}, \textit{name}, \textit{initpop}, \textit{coords}, \textit{geocode}, \textit{values}=\texttt{()})

    \vspace{-1.5ex}

    \rule{\textwidth}{0.5\fboxrule}
    Set initial values for site attributes.

    -name: name of the locality

    -coords: site coordinates.

    -initpop: total population size.

    -geocode: integer id code for site

    -values: Tuple containing adicional values from the sites file

    \vspace{1ex}

    \end{boxedminipage}

    \label{Epigrass:simobj:siteobj:createModel}
    \index{Epigrass \textit{(package)}!Epigrass.simobj \textit{(module)}!Epigrass.simobj.siteobj \textit{(class)}!Epigrass.simobj.siteobj.createModel \textit{(method)}}

    \vspace{0.5ex}

    \begin{boxedminipage}{\textwidth}

    \raggedright \textbf{createModel}(\textit{self}, \textit{init}, \textit{par}, \textit{modtype}=\texttt{'SEIR'}, \textit{name}=\texttt{'model1'}, \textit{v}=\texttt{[]}, \textit{bi}=\texttt{None}, \textit{bp}=\texttt{None})

    \vspace{-1.5ex}

    \rule{\textwidth}{0.5\fboxrule}
    Creates a model of type modtype and defines its initial parameters. 
    init -- initial conditions for the state variables tuple with fractions
    of the total population in each category (state variable). par -- 
    initial values for the parameters. v -- List of extra variables passed 
    in the sites files bi, bp -- dictionaries containing all the inits and 
    parms defined in the .epg model

    \vspace{1ex}

    \end{boxedminipage}

    \label{Epigrass:simobj:siteobj:runModel}
    \index{Epigrass \textit{(package)}!Epigrass.simobj \textit{(module)}!Epigrass.simobj.siteobj \textit{(class)}!Epigrass.simobj.siteobj.runModel \textit{(method)}}

    \vspace{0.5ex}

    \begin{boxedminipage}{\textwidth}

    \raggedright \textbf{runModel}(\textit{self})

    \vspace{-1.5ex}

    \rule{\textwidth}{0.5\fboxrule}
    Iterate the model

    \vspace{1ex}

    \end{boxedminipage}

    \label{Epigrass:simobj:siteobj:vaccinate}
    \index{Epigrass \textit{(package)}!Epigrass.simobj \textit{(module)}!Epigrass.simobj.siteobj \textit{(class)}!Epigrass.simobj.siteobj.vaccinate \textit{(method)}}

    \vspace{0.5ex}

    \begin{boxedminipage}{\textwidth}

    \raggedright \textbf{vaccinate}(\textit{self}, \textit{cov})

    \vspace{-1.5ex}

    \rule{\textwidth}{0.5\fboxrule}
    At time t the population will be vaccinated with coverage cov.

    \vspace{1ex}

    \end{boxedminipage}

    \label{Epigrass:simobj:siteobj:intervention}
    \index{Epigrass \textit{(package)}!Epigrass.simobj \textit{(module)}!Epigrass.simobj.siteobj \textit{(class)}!Epigrass.simobj.siteobj.intervention \textit{(method)}}

    \vspace{0.5ex}

    \begin{boxedminipage}{\textwidth}

    \raggedright \textbf{intervention}(\textit{self}, \textit{par}, \textit{cov}, \textit{efic})

    \vspace{-1.5ex}

    \rule{\textwidth}{0.5\fboxrule}
    From time t on, parameter par is changed to par * (1-cov*efic)

    \vspace{1ex}

    \end{boxedminipage}

    \label{Epigrass:simobj:siteobj:getTheta}
    \index{Epigrass \textit{(package)}!Epigrass.simobj \textit{(module)}!Epigrass.simobj.siteobj \textit{(class)}!Epigrass.simobj.siteobj.getTheta \textit{(method)}}

    \vspace{0.5ex}

    \begin{boxedminipage}{\textwidth}

    \raggedright \textbf{getTheta}(\textit{self}, \textit{npass}, \textit{delay})

    \vspace{-1.5ex}

    \rule{\textwidth}{0.5\fboxrule}
    Returns the number of infected individuals in this site commuting 
    through the edge that called this function.

    npass -- number of individuals leaving the node.

    \vspace{1ex}

    \end{boxedminipage}

    \label{Epigrass:simobj:siteobj:getThetaindex}
    \index{Epigrass \textit{(package)}!Epigrass.simobj \textit{(module)}!Epigrass.simobj.siteobj \textit{(class)}!Epigrass.simobj.siteobj.getThetaindex \textit{(method)}}

    \vspace{0.5ex}

    \begin{boxedminipage}{\textwidth}

    \raggedright \textbf{getThetaindex}(\textit{self})

    \vspace{-1.5ex}

    \rule{\textwidth}{0.5\fboxrule}
    Returns the Theta index. Measures the function of a node, that is the 
    average amount of traffic per intersection. The higher theta is, the 
    greater the load of the network.

    \vspace{1ex}

    \end{boxedminipage}

    \label{Epigrass:simobj:siteobj:receiveTheta}
    \index{Epigrass \textit{(package)}!Epigrass.simobj \textit{(module)}!Epigrass.simobj.siteobj \textit{(class)}!Epigrass.simobj.siteobj.receiveTheta \textit{(method)}}

    \vspace{0.5ex}

    \begin{boxedminipage}{\textwidth}

    \raggedright \textbf{receiveTheta}(\textit{self}, \textit{thetai}, \textit{npass}, \textit{sname})

    \vspace{-1.5ex}

    \rule{\textwidth}{0.5\fboxrule}
    Number of infectious individuals arriving from site i

    \vspace{1ex}

    \end{boxedminipage}

    \label{Epigrass:simobj:siteobj:plotItself}
    \index{Epigrass \textit{(package)}!Epigrass.simobj \textit{(module)}!Epigrass.simobj.siteobj \textit{(class)}!Epigrass.simobj.siteobj.plotItself \textit{(method)}}

    \vspace{0.5ex}

    \begin{boxedminipage}{\textwidth}

    \raggedright \textbf{plotItself}(\textit{self})

    \vspace{-1.5ex}

    \rule{\textwidth}{0.5\fboxrule}
    plot site timeseries

    \vspace{1ex}

    \end{boxedminipage}

    \label{Epigrass:simobj:siteobj:isNode}
    \index{Epigrass \textit{(package)}!Epigrass.simobj \textit{(module)}!Epigrass.simobj.siteobj \textit{(class)}!Epigrass.simobj.siteobj.isNode \textit{(method)}}

    \vspace{0.5ex}

    \begin{boxedminipage}{\textwidth}

    \raggedright \textbf{isNode}(\textit{self})

    \vspace{-1.5ex}

    \rule{\textwidth}{0.5\fboxrule}
    find is given site is a node of a graph

    \vspace{1ex}

    \end{boxedminipage}

    \label{Epigrass:simobj:siteobj:getNeighbors}
    \index{Epigrass \textit{(package)}!Epigrass.simobj \textit{(module)}!Epigrass.simobj.siteobj \textit{(class)}!Epigrass.simobj.siteobj.getNeighbors \textit{(method)}}

    \vspace{0.5ex}

    \begin{boxedminipage}{\textwidth}

    \raggedright \textbf{getNeighbors}(\textit{self})

    \vspace{-1.5ex}

    \rule{\textwidth}{0.5\fboxrule}
    Returns a dictionary of neighbooring sites as keys, and distances as 
    values.

    \vspace{1ex}

    \end{boxedminipage}

    \label{Epigrass:simobj:siteobj:getDistanceFromNeighbor}
    \index{Epigrass \textit{(package)}!Epigrass.simobj \textit{(module)}!Epigrass.simobj.siteobj \textit{(class)}!Epigrass.simobj.siteobj.getDistanceFromNeighbor \textit{(method)}}

    \vspace{0.5ex}

    \begin{boxedminipage}{\textwidth}

    \raggedright \textbf{getDistanceFromNeighbor}(\textit{self}, \textit{neighbor})

    \vspace{-1.5ex}

    \rule{\textwidth}{0.5\fboxrule}
    Returns the distance in Km from a given neighbor. neighbor can be a 
    siteobj object, or a geocode number

    \vspace{1ex}

    \end{boxedminipage}

    \label{Epigrass:simobj:siteobj:getDegree}
    \index{Epigrass \textit{(package)}!Epigrass.simobj \textit{(module)}!Epigrass.simobj.siteobj \textit{(class)}!Epigrass.simobj.siteobj.getDegree \textit{(method)}}

    \vspace{0.5ex}

    \begin{boxedminipage}{\textwidth}

    \raggedright \textbf{getDegree}(\textit{self})

    \vspace{-1.5ex}

    \rule{\textwidth}{0.5\fboxrule}
    Returns the degrees of this site if it is part of a graph. The order 
    (degree) of a node is the number of its attached links and is a simple,
    but effective measure of nodal importance.

    The higher its value, the more a node is important in a graph as many 
    links converge to it. Hub nodes have a high order, while terminal 
    points have an order that can be as low as 1.

    A perfect hub would have its order equal to the summation of all the 
    orders of the other nodes in the graph and a perfect spoke would have 
    an order of 1.

    \vspace{1ex}

    \end{boxedminipage}

    \label{Epigrass:simobj:siteobj:doStats}
    \index{Epigrass \textit{(package)}!Epigrass.simobj \textit{(module)}!Epigrass.simobj.siteobj \textit{(class)}!Epigrass.simobj.siteobj.doStats \textit{(method)}}

    \vspace{0.5ex}

    \begin{boxedminipage}{\textwidth}

    \raggedright \textbf{doStats}(\textit{self})

    \vspace{-1.5ex}

    \rule{\textwidth}{0.5\fboxrule}
    Calculate indices describing the node and return them in a list.

    \vspace{1ex}

    \end{boxedminipage}

    \label{Epigrass:simobj:siteobj:getCentrality}
    \index{Epigrass \textit{(package)}!Epigrass.simobj \textit{(module)}!Epigrass.simobj.siteobj \textit{(class)}!Epigrass.simobj.siteobj.getCentrality \textit{(method)}}

    \vspace{0.5ex}

    \begin{boxedminipage}{\textwidth}

    \raggedright \textbf{getCentrality}(\textit{self})

    \vspace{-1.5ex}

    \rule{\textwidth}{0.5\fboxrule}
    Also known as closeness. A measure of global centrality, is the inverse
    of the sum of the shortest paths to all other nodes in the graph.

    \vspace{1ex}

    \end{boxedminipage}

    \label{Epigrass:simobj:siteobj:getBetweeness}
    \index{Epigrass \textit{(package)}!Epigrass.simobj \textit{(module)}!Epigrass.simobj.siteobj \textit{(class)}!Epigrass.simobj.siteobj.getBetweeness \textit{(method)}}

    \vspace{0.5ex}

    \begin{boxedminipage}{\textwidth}

    \raggedright \textbf{getBetweeness}(\textit{self})

    \vspace{-1.5ex}

    \rule{\textwidth}{0.5\fboxrule}
    Is the number of times any node figures in the the shortest path 
    between any other pair of nodes.

    \vspace{1ex}

    \end{boxedminipage}

    \index{Epigrass \textit{(package)}!Epigrass.simobj \textit{(module)}!Epigrass.simobj.siteobj \textit{(class)}|)}

%%%%%%%%%%%%%%%%%%%%%%%%%%%%%%%%%%%%%%%%%%%%%%%%%%%%%%%%%%%%%%%%%%%%%%%%%%%
%%                           Class Description                           %%
%%%%%%%%%%%%%%%%%%%%%%%%%%%%%%%%%%%%%%%%%%%%%%%%%%%%%%%%%%%%%%%%%%%%%%%%%%%

    \index{Epigrass \textit{(package)}!Epigrass.simobj \textit{(module)}!Epigrass.simobj.popmodels \textit{(class)}|(}
\subsection{Class popmodels}

    \label{Epigrass:simobj:popmodels}
Defines a library of discrete time population models


%%%%%%%%%%%%%%%%%%%%%%%%%%%%%%%%%%%%%%%%%%%%%%%%%%%%%%%%%%%%%%%%%%%%%%%%%%%
%%                                Methods                                %%
%%%%%%%%%%%%%%%%%%%%%%%%%%%%%%%%%%%%%%%%%%%%%%%%%%%%%%%%%%%%%%%%%%%%%%%%%%%

  \subsubsection{Methods}

    \label{Epigrass:simobj:popmodels:__init__}
    \index{Epigrass \textit{(package)}!Epigrass.simobj \textit{(module)}!Epigrass.simobj.popmodels \textit{(class)}!Epigrass.simobj.popmodels.\_\_init\_\_ \textit{(method)}}

    \vspace{0.5ex}

    \begin{boxedminipage}{\textwidth}

    \raggedright \textbf{\_\_init\_\_}(\textit{self}, \textit{parentsite}, \textit{par}, \textit{type}=\texttt{'SIR'}, \textit{v}=\texttt{[]}, \textit{bi}=\texttt{None}, \textit{bp}=\texttt{None})

    \vspace{-1.5ex}

    \rule{\textwidth}{0.5\fboxrule}
    defines which models a given site will use and set variable names 
    accordingly.

    \vspace{1ex}

    \end{boxedminipage}

    \label{Epigrass:simobj:popmodels:selectModel}
    \index{Epigrass \textit{(package)}!Epigrass.simobj \textit{(module)}!Epigrass.simobj.popmodels \textit{(class)}!Epigrass.simobj.popmodels.selectModel \textit{(method)}}

    \vspace{0.5ex}

    \begin{boxedminipage}{\textwidth}

    \raggedright \textbf{selectModel}(\textit{self}, \textit{type})

    \vspace{-1.5ex}

    \rule{\textwidth}{0.5\fboxrule}
    sets the model engine

    \vspace{1ex}

    \end{boxedminipage}

    \label{Epigrass:simobj:popmodels:multipleStep}
    \index{Epigrass \textit{(package)}!Epigrass.simobj \textit{(module)}!Epigrass.simobj.popmodels \textit{(class)}!Epigrass.simobj.popmodels.multipleStep \textit{(method)}}

    \vspace{0.5ex}

    \begin{boxedminipage}{\textwidth}

    \raggedright \textbf{multipleStep}(\textit{self}, \textit{inits}, \textit{par}, \textit{theta}=\texttt{0}, \textit{npass}=\texttt{0}, \textit{modelos}=\texttt{[]})

    \vspace{-1.5ex}

    \rule{\textwidth}{0.5\fboxrule}
\begin{alltt}

Run multiple models on a single site
- Inits and par are a list of lists.
- modelos is a list of of the modeltypes.
\end{alltt}

    \vspace{1ex}

    \end{boxedminipage}

    \label{Epigrass:simobj:popmodels:stepFlu}
    \index{Epigrass \textit{(package)}!Epigrass.simobj \textit{(module)}!Epigrass.simobj.popmodels \textit{(class)}!Epigrass.simobj.popmodels.stepFlu \textit{(method)}}

    \vspace{0.5ex}

    \begin{boxedminipage}{\textwidth}

    \raggedright \textbf{stepFlu}(\textit{self}, \textit{vars}, \textit{par}, \textit{theta}=\texttt{0}, \textit{npass}=\texttt{0})

    \vspace{-1.5ex}

    \rule{\textwidth}{0.5\fboxrule}
    Flu model with classes S,E,I subclinical, I mild, I medium, I serious, 
    deaths

    \vspace{1ex}

    \end{boxedminipage}

    \label{Epigrass:simobj:popmodels:stepSIS}
    \index{Epigrass \textit{(package)}!Epigrass.simobj \textit{(module)}!Epigrass.simobj.popmodels \textit{(class)}!Epigrass.simobj.popmodels.stepSIS \textit{(method)}}

    \vspace{0.5ex}

    \begin{boxedminipage}{\textwidth}

    \raggedright \textbf{stepSIS}(\textit{self}, \textit{inits}, \textit{par}, \textit{theta}=\texttt{0}, \textit{npass}=\texttt{0})

    \vspace{-1.5ex}

    \rule{\textwidth}{0.5\fboxrule}
\begin{alltt}

calculates the model SIS, and return its values (no demographics)
- inits = (E,I,S)
- par = (Beta, alpha, E,r,delta,B, w, p) see docs.
- theta = infectious individuals from neighbor sites
\end{alltt}

    \vspace{1ex}

    \end{boxedminipage}

    \label{Epigrass:simobj:popmodels:stepSIS_s}
    \index{Epigrass \textit{(package)}!Epigrass.simobj \textit{(module)}!Epigrass.simobj.popmodels \textit{(class)}!Epigrass.simobj.popmodels.stepSIS\_s \textit{(method)}}

    \vspace{0.5ex}

    \begin{boxedminipage}{\textwidth}

    \raggedright \textbf{stepSIS\_s}(\textit{self}, \textit{inits}=\texttt{(0,0,0)}, \textit{par}=\texttt{(0.001,1,1,0.5,1,0,0,0)}, \textit{theta}=\texttt{0}, \textit{npass}=\texttt{0}, \textit{dist}=\texttt{'poisson'})

    \vspace{-1.5ex}

    \rule{\textwidth}{0.5\fboxrule}
\begin{alltt}

Defines an stochastic model SIS:
- inits = (E,I,S)
- par = (Beta, alpha, E,r,delta,B,w,p) see docs.
- theta = infectious individuals from neighbor sites
\end{alltt}

    \vspace{1ex}

    \end{boxedminipage}

    \label{Epigrass:simobj:popmodels:stepSIR}
    \index{Epigrass \textit{(package)}!Epigrass.simobj \textit{(module)}!Epigrass.simobj.popmodels \textit{(class)}!Epigrass.simobj.popmodels.stepSIR \textit{(method)}}

    \vspace{0.5ex}

    \begin{boxedminipage}{\textwidth}

    \raggedright \textbf{stepSIR}(\textit{self}, \textit{inits}, \textit{par}, \textit{theta}=\texttt{0}, \textit{npass}=\texttt{0})

    \vspace{-1.5ex}

    \rule{\textwidth}{0.5\fboxrule}
\begin{alltt}

calculates the model SIR, and return its values (no demographics)
- inits = (E,I,S)
- par = (Beta, alpha, E,r,delta,B, w, p) see docs.
- theta = infectious individuals from neighbor sites
\end{alltt}

    \vspace{1ex}

    \end{boxedminipage}

    \label{Epigrass:simobj:popmodels:stepSIR_s}
    \index{Epigrass \textit{(package)}!Epigrass.simobj \textit{(module)}!Epigrass.simobj.popmodels \textit{(class)}!Epigrass.simobj.popmodels.stepSIR\_s \textit{(method)}}

    \vspace{0.5ex}

    \begin{boxedminipage}{\textwidth}

    \raggedright \textbf{stepSIR\_s}(\textit{self}, \textit{inits}=\texttt{(0,0,0)}, \textit{par}=\texttt{(0.001,1,1,0.5,1,0,0,0)}, \textit{theta}=\texttt{0}, \textit{npass}=\texttt{0}, \textit{dist}=\texttt{'poisson'})

    \vspace{-1.5ex}

    \rule{\textwidth}{0.5\fboxrule}
\begin{alltt}

Defines an stochastic model SIR:
- inits = (E,I,S)
- par = (Beta, alpha, E,r,delta,B,w,p) see docs.
- theta = infectious individuals from neighbor sites
\end{alltt}

    \vspace{1ex}

    \end{boxedminipage}

    \label{Epigrass:simobj:popmodels:stepSEIS}
    \index{Epigrass \textit{(package)}!Epigrass.simobj \textit{(module)}!Epigrass.simobj.popmodels \textit{(class)}!Epigrass.simobj.popmodels.stepSEIS \textit{(method)}}

    \vspace{0.5ex}

    \begin{boxedminipage}{\textwidth}

    \raggedright \textbf{stepSEIS}(\textit{self}, \textit{inits}=\texttt{(0,0,0)}, \textit{par}=\texttt{(0.001,1,1,0.5,1,0,0,0)}, \textit{theta}=\texttt{0}, \textit{npass}=\texttt{0})

    \vspace{-1.5ex}

    \rule{\textwidth}{0.5\fboxrule}
\begin{alltt}

Defines the model SEIS:
- inits = (E,I,S)
- par = (Beta, alpha, E,r,delta,B,w,p) see docs.
- theta = infectious individuals from neighbor sites
\end{alltt}

    \vspace{1ex}

    \end{boxedminipage}

    \label{Epigrass:simobj:popmodels:stepSEIS_s}
    \index{Epigrass \textit{(package)}!Epigrass.simobj \textit{(module)}!Epigrass.simobj.popmodels \textit{(class)}!Epigrass.simobj.popmodels.stepSEIS\_s \textit{(method)}}

    \vspace{0.5ex}

    \begin{boxedminipage}{\textwidth}

    \raggedright \textbf{stepSEIS\_s}(\textit{self}, \textit{inits}=\texttt{(0,0,0)}, \textit{par}=\texttt{(0.001,1,1,0.5,1,0,0,0)}, \textit{theta}=\texttt{0}, \textit{npass}=\texttt{0}, \textit{dist}=\texttt{'poisson'})

    \vspace{-1.5ex}

    \rule{\textwidth}{0.5\fboxrule}
\begin{alltt}

Defines an stochastic model SEIS:
- inits = (E,I,S)
- par = (Beta, alpha, E,r,delta,B,w,p) see docs.
- theta = infectious individuals from neighbor sites
\end{alltt}

    \vspace{1ex}

    \end{boxedminipage}

    \label{Epigrass:simobj:popmodels:stepSEIR}
    \index{Epigrass \textit{(package)}!Epigrass.simobj \textit{(module)}!Epigrass.simobj.popmodels \textit{(class)}!Epigrass.simobj.popmodels.stepSEIR \textit{(method)}}

    \vspace{0.5ex}

    \begin{boxedminipage}{\textwidth}

    \raggedright \textbf{stepSEIR}(\textit{self}, \textit{inits}=\texttt{(0,0,0)}, \textit{par}=\texttt{(0.001,1,1,0.5,1,0,0,0)}, \textit{theta}=\texttt{0}, \textit{npass}=\texttt{0})

    \vspace{-1.5ex}

    \rule{\textwidth}{0.5\fboxrule}
\begin{alltt}

Defines the model SEIR:
- inits = (E,I,S)
- par = (Beta, alpha, E,r,delta,B,w,p) see docs.
- theta = infectious individuals from neighbor sites
\end{alltt}

    \vspace{1ex}

    \end{boxedminipage}

    \label{Epigrass:simobj:popmodels:stepSEIR_s}
    \index{Epigrass \textit{(package)}!Epigrass.simobj \textit{(module)}!Epigrass.simobj.popmodels \textit{(class)}!Epigrass.simobj.popmodels.stepSEIR\_s \textit{(method)}}

    \vspace{0.5ex}

    \begin{boxedminipage}{\textwidth}

    \raggedright \textbf{stepSEIR\_s}(\textit{self}, \textit{inits}=\texttt{(0,0,0)}, \textit{par}=\texttt{(0.001,1,1,0.5,1,0,0,0)}, \textit{theta}=\texttt{0}, \textit{npass}=\texttt{0}, \textit{dist}=\texttt{'poisson'})

    \vspace{-1.5ex}

    \rule{\textwidth}{0.5\fboxrule}
\begin{alltt}

Defines an stochastic model SEIR:
- inits = (E,I,S)
- par = (Beta, alpha, E,r,delta,B,w,p) see docs.
- theta = infectious individuals from neighbor sites
\end{alltt}

    \vspace{1ex}

    \end{boxedminipage}

    \label{Epigrass:simobj:popmodels:stepSIpRpS}
    \index{Epigrass \textit{(package)}!Epigrass.simobj \textit{(module)}!Epigrass.simobj.popmodels \textit{(class)}!Epigrass.simobj.popmodels.stepSIpRpS \textit{(method)}}

    \vspace{0.5ex}

    \begin{boxedminipage}{\textwidth}

    \raggedright \textbf{stepSIpRpS}(\textit{self}, \textit{inits}, \textit{par}, \textit{theta}=\texttt{0}, \textit{npass}=\texttt{0})

    \vspace{-1.5ex}

    \rule{\textwidth}{0.5\fboxrule}
\begin{alltt}

calculates the model SIpRpS, and return its values (no demographics)
- inits = (E,I,S)
- par = (Beta, alpha, E,r,delta,B, w, p) see docs.
- theta = infectious individuals from neighbor sites
\end{alltt}

    \vspace{1ex}

    \end{boxedminipage}

    \label{Epigrass:simobj:popmodels:stepSIpRpS_s}
    \index{Epigrass \textit{(package)}!Epigrass.simobj \textit{(module)}!Epigrass.simobj.popmodels \textit{(class)}!Epigrass.simobj.popmodels.stepSIpRpS\_s \textit{(method)}}

    \vspace{0.5ex}

    \begin{boxedminipage}{\textwidth}

    \raggedright \textbf{stepSIpRpS\_s}(\textit{self}, \textit{inits}=\texttt{(0,0,0)}, \textit{par}=\texttt{(0.001,1,1,0.5,1,0,0,0)}, \textit{theta}=\texttt{0}, \textit{npass}=\texttt{0}, \textit{dist}=\texttt{'poisson'})

    \vspace{-1.5ex}

    \rule{\textwidth}{0.5\fboxrule}
\begin{alltt}

Defines an stochastic model SIpRpS:
- inits = (E,I,S)
- par = (Beta, alpha, E,r,delta,B,w,p) see docs.
- theta = infectious individuals from neighbor sites
\end{alltt}

    \vspace{1ex}

    \end{boxedminipage}

    \label{Epigrass:simobj:popmodels:stepSEIpRpS}
    \index{Epigrass \textit{(package)}!Epigrass.simobj \textit{(module)}!Epigrass.simobj.popmodels \textit{(class)}!Epigrass.simobj.popmodels.stepSEIpRpS \textit{(method)}}

    \vspace{0.5ex}

    \begin{boxedminipage}{\textwidth}

    \raggedright \textbf{stepSEIpRpS}(\textit{self}, \textit{inits}=\texttt{(0,0,0)}, \textit{par}=\texttt{(0.001,1,1,0.5,1,0,0,0)}, \textit{theta}=\texttt{0}, \textit{npass}=\texttt{0})

    \vspace{-1.5ex}

    \rule{\textwidth}{0.5\fboxrule}
\begin{alltt}

Defines the model SEIpRpS:
- inits = (E,I,S)
- par = (Beta, alpha, E,r,delta,B,w,p) see docs.
- theta = infectious individuals from neighbor sites
\end{alltt}

    \vspace{1ex}

    \end{boxedminipage}

    \label{Epigrass:simobj:popmodels:stepSEIpRpS_s}
    \index{Epigrass \textit{(package)}!Epigrass.simobj \textit{(module)}!Epigrass.simobj.popmodels \textit{(class)}!Epigrass.simobj.popmodels.stepSEIpRpS\_s \textit{(method)}}

    \vspace{0.5ex}

    \begin{boxedminipage}{\textwidth}

    \raggedright \textbf{stepSEIpRpS\_s}(\textit{self}, \textit{inits}=\texttt{(0,0,0)}, \textit{par}=\texttt{(0.001,1,1,0.5,1,0,0,0)}, \textit{theta}=\texttt{0}, \textit{npass}=\texttt{0}, \textit{dist}=\texttt{'poisson'})

    \vspace{-1.5ex}

    \rule{\textwidth}{0.5\fboxrule}
\begin{alltt}

Defines an stochastic model SEIpRpS:
- inits = (E,I,S)
- par = (Beta, alpha, E,r,delta,B,w,p) see docs.
- theta = infectious individuals from neighbor sites
\end{alltt}

    \vspace{1ex}

    \end{boxedminipage}

    \label{Epigrass:simobj:popmodels:stepSIpR}
    \index{Epigrass \textit{(package)}!Epigrass.simobj \textit{(module)}!Epigrass.simobj.popmodels \textit{(class)}!Epigrass.simobj.popmodels.stepSIpR \textit{(method)}}

    \vspace{0.5ex}

    \begin{boxedminipage}{\textwidth}

    \raggedright \textbf{stepSIpR}(\textit{self}, \textit{inits}, \textit{par}, \textit{theta}=\texttt{0}, \textit{npass}=\texttt{0})

    \vspace{-1.5ex}

    \rule{\textwidth}{0.5\fboxrule}
\begin{alltt}

calculates the model SIpR, and return its values (no demographics)
- inits = (E,I,S)
- par = (Beta, alpha, E,r,delta,B, w, p) see docs.
- theta = infectious individuals from neighbor sites
\end{alltt}

    \vspace{1ex}

    \end{boxedminipage}

    \label{Epigrass:simobj:popmodels:stepSIpR_s}
    \index{Epigrass \textit{(package)}!Epigrass.simobj \textit{(module)}!Epigrass.simobj.popmodels \textit{(class)}!Epigrass.simobj.popmodels.stepSIpR\_s \textit{(method)}}

    \vspace{0.5ex}

    \begin{boxedminipage}{\textwidth}

    \raggedright \textbf{stepSIpR\_s}(\textit{self}, \textit{inits}=\texttt{(0,0,0)}, \textit{par}=\texttt{(0.001,1,1,0.5,1,0,0,0)}, \textit{theta}=\texttt{0}, \textit{npass}=\texttt{0}, \textit{dist}=\texttt{'poisson'})

    \vspace{-1.5ex}

    \rule{\textwidth}{0.5\fboxrule}
\begin{alltt}

Defines an stochastic model SIpRs:
- inits = (E,I,S)
- par = (Beta, alpha, E,r,delta,B,w,p) see docs.
- theta = infectious individuals from neighbor sites
\end{alltt}

    \vspace{1ex}

    \end{boxedminipage}

    \label{Epigrass:simobj:popmodels:stepSEIpR}
    \index{Epigrass \textit{(package)}!Epigrass.simobj \textit{(module)}!Epigrass.simobj.popmodels \textit{(class)}!Epigrass.simobj.popmodels.stepSEIpR \textit{(method)}}

    \vspace{0.5ex}

    \begin{boxedminipage}{\textwidth}

    \raggedright \textbf{stepSEIpR}(\textit{self}, \textit{inits}, \textit{par}, \textit{theta}=\texttt{0}, \textit{npass}=\texttt{0})

    \vspace{-1.5ex}

    \rule{\textwidth}{0.5\fboxrule}
\begin{alltt}

calculates the model SEIpR, and return its values (no demographics)
- inits = (E,I,S)
- par = (Beta, alpha, E,r,delta,B, w, p) see docs.
- theta = infectious individuals from neighbor sites
\end{alltt}

    \vspace{1ex}

    \end{boxedminipage}

    \label{Epigrass:simobj:popmodels:stepSEIpR_s}
    \index{Epigrass \textit{(package)}!Epigrass.simobj \textit{(module)}!Epigrass.simobj.popmodels \textit{(class)}!Epigrass.simobj.popmodels.stepSEIpR\_s \textit{(method)}}

    \vspace{0.5ex}

    \begin{boxedminipage}{\textwidth}

    \raggedright \textbf{stepSEIpR\_s}(\textit{self}, \textit{inits}=\texttt{(0,0,0)}, \textit{par}=\texttt{(0.001,1,1,0.5,1,0,0,0)}, \textit{theta}=\texttt{0}, \textit{npass}=\texttt{0}, \textit{dist}=\texttt{'poisson'})

    \vspace{-1.5ex}

    \rule{\textwidth}{0.5\fboxrule}
\begin{alltt}

Defines an stochastic model SEIpRs:
- inits = (E,I,S)
- par = (Beta, alpha, E,r,delta,B,w,p) see docs.
- theta = infectious individuals from neighbor sites
\end{alltt}

    \vspace{1ex}

    \end{boxedminipage}

    \label{Epigrass:simobj:popmodels:stepSIRS}
    \index{Epigrass \textit{(package)}!Epigrass.simobj \textit{(module)}!Epigrass.simobj.popmodels \textit{(class)}!Epigrass.simobj.popmodels.stepSIRS \textit{(method)}}

    \vspace{0.5ex}

    \begin{boxedminipage}{\textwidth}

    \raggedright \textbf{stepSIRS}(\textit{self}, \textit{inits}, \textit{par}, \textit{theta}=\texttt{0}, \textit{npass}=\texttt{0})

    \vspace{-1.5ex}

    \rule{\textwidth}{0.5\fboxrule}
\begin{alltt}

calculates the model SIRS, and return its values (no demographics)
- inits = (E,I,S)
- par = (Beta, alpha, E,r,delta,B, w, p) see docs.
- theta = infectious individuals from neighbor sites
\end{alltt}

    \vspace{1ex}

    \end{boxedminipage}

    \label{Epigrass:simobj:popmodels:stepSIRS_s}
    \index{Epigrass \textit{(package)}!Epigrass.simobj \textit{(module)}!Epigrass.simobj.popmodels \textit{(class)}!Epigrass.simobj.popmodels.stepSIRS\_s \textit{(method)}}

    \vspace{0.5ex}

    \begin{boxedminipage}{\textwidth}

    \raggedright \textbf{stepSIRS\_s}(\textit{self}, \textit{inits}=\texttt{(0,0,0)}, \textit{par}=\texttt{(0.001,1,1,0.5,1,0,0,0)}, \textit{theta}=\texttt{0}, \textit{npass}=\texttt{0}, \textit{dist}=\texttt{'poisson'})

    \vspace{-1.5ex}

    \rule{\textwidth}{0.5\fboxrule}
\begin{alltt}

Defines an stochastic model SIR:
- inits = (E,I,S)
- par = (Beta, alpha, E,r,delta,B,w,p) see docs.
- theta = infectious individuals from neighbor sites
\end{alltt}

    \vspace{1ex}

    \end{boxedminipage}

    \index{Epigrass \textit{(package)}!Epigrass.simobj \textit{(module)}!Epigrass.simobj.popmodels \textit{(class)}|)}

%%%%%%%%%%%%%%%%%%%%%%%%%%%%%%%%%%%%%%%%%%%%%%%%%%%%%%%%%%%%%%%%%%%%%%%%%%%
%%                           Class Description                           %%
%%%%%%%%%%%%%%%%%%%%%%%%%%%%%%%%%%%%%%%%%%%%%%%%%%%%%%%%%%%%%%%%%%%%%%%%%%%

    \index{Epigrass \textit{(package)}!Epigrass.simobj \textit{(module)}!Epigrass.simobj.edge \textit{(class)}|(}
\subsection{Class edge}

    \label{Epigrass:simobj:edge}
Defines an edge connecting two nodes (node source to node dest). with 
attributes given by value.


%%%%%%%%%%%%%%%%%%%%%%%%%%%%%%%%%%%%%%%%%%%%%%%%%%%%%%%%%%%%%%%%%%%%%%%%%%%
%%                                Methods                                %%
%%%%%%%%%%%%%%%%%%%%%%%%%%%%%%%%%%%%%%%%%%%%%%%%%%%%%%%%%%%%%%%%%%%%%%%%%%%

  \subsubsection{Methods}

    \label{Epigrass:simobj:edge:__init__}
    \index{Epigrass \textit{(package)}!Epigrass.simobj \textit{(module)}!Epigrass.simobj.edge \textit{(class)}!Epigrass.simobj.edge.\_\_init\_\_ \textit{(method)}}

    \vspace{0.5ex}

    \begin{boxedminipage}{\textwidth}

    \raggedright \textbf{\_\_init\_\_}(\textit{self}, \textit{source}, \textit{dest}, \textit{fmig}=\texttt{0}, \textit{bmig}=\texttt{0}, \textit{Leng}=\texttt{0})

    \vspace{-1.5ex}

    \rule{\textwidth}{0.5\fboxrule}
    Main attributes of *Edge*.

    source -- Source site object.

    dest -- Destination site object.

    fmig -- forward migration rate in number of indiv./day.

    bmig -- backward migration rate in number of indiv./day.

    Leng -- Length in kilometers of this route

    \vspace{1ex}

    \end{boxedminipage}

    \label{Epigrass:simobj:edge:calcDelay}
    \index{Epigrass \textit{(package)}!Epigrass.simobj \textit{(module)}!Epigrass.simobj.edge \textit{(class)}!Epigrass.simobj.edge.calcDelay \textit{(method)}}

    \vspace{0.5ex}

    \begin{boxedminipage}{\textwidth}

    \raggedright \textbf{calcDelay}(\textit{self})

    \vspace{-1.5ex}

    \rule{\textwidth}{0.5\fboxrule}
    calculate the Transportation delay given the speed and length.

    \vspace{1ex}

    \end{boxedminipage}

    \label{Epigrass:simobj:edge:transportStoD}
    \index{Epigrass \textit{(package)}!Epigrass.simobj \textit{(module)}!Epigrass.simobj.edge \textit{(class)}!Epigrass.simobj.edge.transportStoD \textit{(method)}}

    \vspace{0.5ex}

    \begin{boxedminipage}{\textwidth}

    \raggedright \textbf{transportStoD}(\textit{self})

    \vspace{-1.5ex}

    \rule{\textwidth}{0.5\fboxrule}
    Get infectious individuals commuting from source node and inform them 
    to destination

    \vspace{1ex}

    \end{boxedminipage}

    \label{Epigrass:simobj:edge:transportDtoS}
    \index{Epigrass \textit{(package)}!Epigrass.simobj \textit{(module)}!Epigrass.simobj.edge \textit{(class)}!Epigrass.simobj.edge.transportDtoS \textit{(method)}}

    \vspace{0.5ex}

    \begin{boxedminipage}{\textwidth}

    \raggedright \textbf{transportDtoS}(\textit{self})

    \vspace{-1.5ex}

    \rule{\textwidth}{0.5\fboxrule}
    Get infectious individuals commuting from destination node and inform 
    them to source

    \vspace{1ex}

    \end{boxedminipage}

    \index{Epigrass \textit{(package)}!Epigrass.simobj \textit{(module)}!Epigrass.simobj.edge \textit{(class)}|)}

%%%%%%%%%%%%%%%%%%%%%%%%%%%%%%%%%%%%%%%%%%%%%%%%%%%%%%%%%%%%%%%%%%%%%%%%%%%
%%                           Class Description                           %%
%%%%%%%%%%%%%%%%%%%%%%%%%%%%%%%%%%%%%%%%%%%%%%%%%%%%%%%%%%%%%%%%%%%%%%%%%%%

    \index{Epigrass \textit{(package)}!Epigrass.simobj \textit{(module)}!Epigrass.simobj.graph \textit{(class)}|(}
\subsection{Class graph}

    \label{Epigrass:simobj:graph}
Defines a graph with sites and edges


%%%%%%%%%%%%%%%%%%%%%%%%%%%%%%%%%%%%%%%%%%%%%%%%%%%%%%%%%%%%%%%%%%%%%%%%%%%
%%                                Methods                                %%
%%%%%%%%%%%%%%%%%%%%%%%%%%%%%%%%%%%%%%%%%%%%%%%%%%%%%%%%%%%%%%%%%%%%%%%%%%%

  \subsubsection{Methods}

    \label{Epigrass:simobj:graph:__init__}
    \index{Epigrass \textit{(package)}!Epigrass.simobj \textit{(module)}!Epigrass.simobj.graph \textit{(class)}!Epigrass.simobj.graph.\_\_init\_\_ \textit{(method)}}

    \vspace{0.5ex}

    \begin{boxedminipage}{\textwidth}

    \raggedright \textbf{\_\_init\_\_}(\textit{self}, \textit{graph\_name}, \textit{digraph}=\texttt{0})

    \end{boxedminipage}

    \label{Epigrass:simobj:graph:addSite}
    \index{Epigrass \textit{(package)}!Epigrass.simobj \textit{(module)}!Epigrass.simobj.graph \textit{(class)}!Epigrass.simobj.graph.addSite \textit{(method)}}

    \vspace{0.5ex}

    \begin{boxedminipage}{\textwidth}

    \raggedright \textbf{addSite}(\textit{self}, \textit{sitio})

    \vspace{-1.5ex}

    \rule{\textwidth}{0.5\fboxrule}
    Adds a site object to the graph. It takes a siteobj object as its only 
    argument and returns None.

    \vspace{1ex}

    \end{boxedminipage}

    \label{Epigrass:simobj:graph:dijkstra}
    \index{Epigrass \textit{(package)}!Epigrass.simobj \textit{(module)}!Epigrass.simobj.graph \textit{(class)}!Epigrass.simobj.graph.dijkstra \textit{(method)}}

    \vspace{0.5ex}

    \begin{boxedminipage}{\textwidth}

    \raggedright \textbf{dijkstra}(\textit{self}, \textit{G}, \textit{start}, \textit{end}=\texttt{None})

    \vspace{-1.5ex}

    \rule{\textwidth}{0.5\fboxrule}
    Find shortest paths from the start vertex to all vertices nearer than 
    or equal to the end.

    The input graph G is assumed to have the following representation: A 
    vertex can be any object that can be used as an index into a 
    dictionary.  G is a dictionary, indexed by vertices.  For any vertex v,
    G[v] is itself a dictionary, indexed by the neighbors of v.  For any 
    edge v-{\textgreater}w, G[v][w] is the length of the edge.  This is 
    related to the representation in 
    {\textless}http://www.python.org/doc/essays/graphs.html{\textgreater} 
    where Guido van Rossum suggests representing graphs as dictionaries 
    mapping vertices to lists of neighbors, however dictionaries of edges 
    have many advantages over lists: they can store extra information 
    (here, the lengths), they support fast existence tests, and they allow 
    easy modification of the graph by edge insertion and removal.  Such 
    modifications are not needed here but are important in other graph 
    algorithms. Since dictionaries obey iterator protocol, a graph 
    represented as described here could be handed without modification to 
    an algorithm using Guido's representation.

    Of course, G and G[v] need not be Python dict objects; they can be any 
    other object that obeys dict protocol, for instance a wrapper in which 
    vertices are URLs and a call to G[v] loads the web page and finds its 
    links.

    The output is a pair (D,P) where D[v] is the distance from start to v 
    and P[v] is the predecessor of v along the shortest path from s to v.

    Dijkstra's algorithm is only guaranteed to work correctly when all edge
    lengths are positive. This code does not verify this property for all 
    edges (only the edges seen before the end vertex is reached), but will 
    correctly compute shortest paths even for some graphs with negative 
    edges, and will raise an exception if it discovers that a negative edge
    has caused it to make a mistake.

    \vspace{1ex}

    \end{boxedminipage}

    \label{Epigrass:simobj:graph:getSite}
    \index{Epigrass \textit{(package)}!Epigrass.simobj \textit{(module)}!Epigrass.simobj.graph \textit{(class)}!Epigrass.simobj.graph.getSite \textit{(method)}}

    \vspace{0.5ex}

    \begin{boxedminipage}{\textwidth}

    \raggedright \textbf{getSite}(\textit{self}, \textit{name})

    \vspace{-1.5ex}

    \rule{\textwidth}{0.5\fboxrule}
    Retrieved a site from the graph.

    Given a site's name the corresponding Siteobj instance will be 
    returned.

    If multiple sites exist with that name, a list of Siteobj instances is 
    returned.

    If only one site exists, the instance is returned. None is returned 
    otherwise.

    \vspace{1ex}

    \end{boxedminipage}

    \label{Epigrass:simobj:graph:addEdge}
    \index{Epigrass \textit{(package)}!Epigrass.simobj \textit{(module)}!Epigrass.simobj.graph \textit{(class)}!Epigrass.simobj.graph.addEdge \textit{(method)}}

    \vspace{0.5ex}

    \begin{boxedminipage}{\textwidth}

    \raggedright \textbf{addEdge}(\textit{self}, \textit{graph\_edge})

    \vspace{-1.5ex}

    \rule{\textwidth}{0.5\fboxrule}
    Adds an edge object to the graph.

    It takes a edge object as its only argument and returns None.

    \vspace{1ex}

    \end{boxedminipage}

    \label{Epigrass:simobj:graph:getGraphdict}
    \index{Epigrass \textit{(package)}!Epigrass.simobj \textit{(module)}!Epigrass.simobj.graph \textit{(class)}!Epigrass.simobj.graph.getGraphdict \textit{(method)}}

    \vspace{0.5ex}

    \begin{boxedminipage}{\textwidth}

    \raggedright \textbf{getGraphdict}(\textit{self})

    \vspace{-1.5ex}

    \rule{\textwidth}{0.5\fboxrule}
    Generates a dictionary of the graph for use in the shortest path 
    function.

    \vspace{1ex}

    \end{boxedminipage}

    \label{Epigrass:simobj:graph:getEdge}
    \index{Epigrass \textit{(package)}!Epigrass.simobj \textit{(module)}!Epigrass.simobj.graph \textit{(class)}!Epigrass.simobj.graph.getEdge \textit{(method)}}

    \vspace{0.5ex}

    \begin{boxedminipage}{\textwidth}

    \raggedright \textbf{getEdge}(\textit{self}, \textit{src}, \textit{dst})

    \vspace{-1.5ex}

    \rule{\textwidth}{0.5\fboxrule}
    Retrieved an edge from the graph.

    Given an edge's source and destination the corresponding Edge instance 
    will be returned.

    If multiple edges exist with that source and destination, a list of 
    Edge instances is returned.

    If only one edge exists, the instance is returned. None is returned 
    otherwise.

    \vspace{1ex}

    \end{boxedminipage}

    \label{Epigrass:simobj:graph:getSiteNames}
    \index{Epigrass \textit{(package)}!Epigrass.simobj \textit{(module)}!Epigrass.simobj.graph \textit{(class)}!Epigrass.simobj.graph.getSiteNames \textit{(method)}}

    \vspace{0.5ex}

    \begin{boxedminipage}{\textwidth}

    \raggedright \textbf{getSiteNames}(\textit{self})

    \vspace{-1.5ex}

    \rule{\textwidth}{0.5\fboxrule}
    returns list of site names for a given graph.

    \vspace{1ex}

    \end{boxedminipage}

    \label{Epigrass:simobj:graph:getCycles}
    \index{Epigrass \textit{(package)}!Epigrass.simobj \textit{(module)}!Epigrass.simobj.graph \textit{(class)}!Epigrass.simobj.graph.getCycles \textit{(method)}}

    \vspace{0.5ex}

    \begin{boxedminipage}{\textwidth}

    \raggedright \textbf{getCycles}(\textit{self})

    \vspace{-1.5ex}

    \rule{\textwidth}{0.5\fboxrule}
    The maximum number of independent cycles in a graph.

    This number (u) is estimated by knowing the number of nodes (v), links 
    (e) and of sub-graphs (p); u = e-v+p.

    Trees and simple networks will have a value of 0 since they have no 
    cycles.

    The more complex a network is, the higher the value of u, so it can be 
    used as an indicator of the level of development of a transport system.

    \vspace{1ex}

    \end{boxedminipage}

    \label{Epigrass:simobj:graph:shortestPath}
    \index{Epigrass \textit{(package)}!Epigrass.simobj \textit{(module)}!Epigrass.simobj.graph \textit{(class)}!Epigrass.simobj.graph.shortestPath \textit{(method)}}

    \vspace{0.5ex}

    \begin{boxedminipage}{\textwidth}

    \raggedright \textbf{shortestPath}(\textit{self}, \textit{G}, \textit{start}, \textit{end})

    \vspace{-1.5ex}

    \rule{\textwidth}{0.5\fboxrule}
    Find a single shortest path from the given start node to the given end 
    node. The input has the same conventions as self.dijkstra(). 'G' is the
    graph's dictionary self.graphdict. 'start' and 'end' are site objects. 
    The output is a list of the vertices in order along the shortest path.

    \vspace{1ex}

    \end{boxedminipage}

    \label{Epigrass:simobj:graph:clearVisual}
    \index{Epigrass \textit{(package)}!Epigrass.simobj \textit{(module)}!Epigrass.simobj.graph \textit{(class)}!Epigrass.simobj.graph.clearVisual \textit{(method)}}

    \vspace{0.5ex}

    \begin{boxedminipage}{\textwidth}

    \raggedright \textbf{clearVisual}(\textit{self})

    \vspace{-1.5ex}

    \rule{\textwidth}{0.5\fboxrule}
    Clear the visual graph display

    \vspace{1ex}

    \end{boxedminipage}

    \label{Epigrass:simobj:graph:viewGraph}
    \index{Epigrass \textit{(package)}!Epigrass.simobj \textit{(module)}!Epigrass.simobj.graph \textit{(class)}!Epigrass.simobj.graph.viewGraph \textit{(method)}}

    \vspace{0.5ex}

    \begin{boxedminipage}{\textwidth}

    \raggedright \textbf{viewGraph}(\textit{self}, \textit{mapa}=\texttt{'limites.txt'})

    \vspace{-1.5ex}

    \rule{\textwidth}{0.5\fboxrule}
    Starts the Vpython display of the graph.

    \vspace{1ex}

    \end{boxedminipage}

    \label{Epigrass:simobj:graph:lightGRNode}
    \index{Epigrass \textit{(package)}!Epigrass.simobj \textit{(module)}!Epigrass.simobj.graph \textit{(class)}!Epigrass.simobj.graph.lightGRNode \textit{(method)}}

    \vspace{0.5ex}

    \begin{boxedminipage}{\textwidth}

    \raggedright \textbf{lightGRNode}(\textit{self}, \textit{node}, \textit{color}=\texttt{'r'})

    \vspace{-1.5ex}

    \rule{\textwidth}{0.5\fboxrule}
    Paints red the sphere corresponding to the node in the visual display

    \vspace{1ex}

    \end{boxedminipage}

    \label{Epigrass:simobj:graph:drawGraph}
    \index{Epigrass \textit{(package)}!Epigrass.simobj \textit{(module)}!Epigrass.simobj.graph \textit{(class)}!Epigrass.simobj.graph.drawGraph \textit{(method)}}

    \vspace{0.5ex}

    \begin{boxedminipage}{\textwidth}

    \raggedright \textbf{drawGraph}(\textit{self})

    \vspace{-1.5ex}

    \rule{\textwidth}{0.5\fboxrule}
    Draws the network using pylab

    \vspace{1ex}

    \end{boxedminipage}

    \label{Epigrass:simobj:graph:drawGraphR}
    \index{Epigrass \textit{(package)}!Epigrass.simobj \textit{(module)}!Epigrass.simobj.graph \textit{(class)}!Epigrass.simobj.graph.drawGraphR \textit{(method)}}

    \vspace{0.5ex}

    \begin{boxedminipage}{\textwidth}

    \raggedright \textbf{drawGraphR}(\textit{self})

    \vspace{-1.5ex}

    \rule{\textwidth}{0.5\fboxrule}
    Draws the network using R

    \vspace{1ex}

    \end{boxedminipage}

    \label{Epigrass:simobj:graph:getAllPairs}
    \index{Epigrass \textit{(package)}!Epigrass.simobj \textit{(module)}!Epigrass.simobj.graph \textit{(class)}!Epigrass.simobj.graph.getAllPairs \textit{(method)}}

    \vspace{0.5ex}

    \begin{boxedminipage}{\textwidth}

    \raggedright \textbf{getAllPairs}(\textit{self})

    \vspace{-1.5ex}

    \rule{\textwidth}{0.5\fboxrule}
    Returns a distance matrix for the graph nodes where the distance is the
    shortest path. Creates another distance matrix where the distances are 
    the lengths of the paths.

    \vspace{1ex}

    \end{boxedminipage}

    \label{Epigrass:simobj:graph:getShortestPathLength}
    \index{Epigrass \textit{(package)}!Epigrass.simobj \textit{(module)}!Epigrass.simobj.graph \textit{(class)}!Epigrass.simobj.graph.getShortestPathLength \textit{(method)}}

    \vspace{0.5ex}

    \begin{boxedminipage}{\textwidth}

    \raggedright \textbf{getShortestPathLength}(\textit{self}, \textit{origin}, \textit{sp})

    \vspace{-1.5ex}

    \rule{\textwidth}{0.5\fboxrule}
    Returns sp Length

    \vspace{1ex}

    \end{boxedminipage}

    \label{Epigrass:simobj:graph:getConnMatrix}
    \index{Epigrass \textit{(package)}!Epigrass.simobj \textit{(module)}!Epigrass.simobj.graph \textit{(class)}!Epigrass.simobj.graph.getConnMatrix \textit{(method)}}

    \vspace{0.5ex}

    \begin{boxedminipage}{\textwidth}

    \raggedright \textbf{getConnMatrix}(\textit{self})

    \vspace{-1.5ex}

    \rule{\textwidth}{0.5\fboxrule}
    The most basic measure of accessibility involves network connectivity 
    where a network is represented as a  connectivity matrix (C1), which 
    expresses the connectivity of each node with its adjacent nodes.

    The number of columns and rows in this matrix is equal to the number of
    nodes in the network and a value of 1 is given for each cell where this
    is a connected pair and a value of 0 for each cell where there is an 
    unconnected pair. The summation of this matrix provides a very basic 
    measure of accessibility, also known as the degree of a node.

    \vspace{1ex}

    \end{boxedminipage}

    \label{Epigrass:simobj:graph:getWienerD}
    \index{Epigrass \textit{(package)}!Epigrass.simobj \textit{(module)}!Epigrass.simobj.graph \textit{(class)}!Epigrass.simobj.graph.getWienerD \textit{(method)}}

    \vspace{0.5ex}

    \begin{boxedminipage}{\textwidth}

    \raggedright \textbf{getWienerD}(\textit{self})

    \vspace{-1.5ex}

    \rule{\textwidth}{0.5\fboxrule}
    Returns the Wiener distance for a graph.

    \vspace{1ex}

    \end{boxedminipage}

    \label{Epigrass:simobj:graph:getMeanD}
    \index{Epigrass \textit{(package)}!Epigrass.simobj \textit{(module)}!Epigrass.simobj.graph \textit{(class)}!Epigrass.simobj.graph.getMeanD \textit{(method)}}

    \vspace{0.5ex}

    \begin{boxedminipage}{\textwidth}

    \raggedright \textbf{getMeanD}(\textit{self})

    \vspace{-1.5ex}

    \rule{\textwidth}{0.5\fboxrule}
    Returns the mean distance for a graph.

    \vspace{1ex}

    \end{boxedminipage}

    \label{Epigrass:simobj:graph:getDiameter}
    \index{Epigrass \textit{(package)}!Epigrass.simobj \textit{(module)}!Epigrass.simobj.graph \textit{(class)}!Epigrass.simobj.graph.getDiameter \textit{(method)}}

    \vspace{0.5ex}

    \begin{boxedminipage}{\textwidth}

    \raggedright \textbf{getDiameter}(\textit{self})

    \vspace{-1.5ex}

    \rule{\textwidth}{0.5\fboxrule}
    Returns the diameter of the graph: longest shortest path.

    \vspace{1ex}

    \end{boxedminipage}

    \label{Epigrass:simobj:graph:getIotaindex}
    \index{Epigrass \textit{(package)}!Epigrass.simobj \textit{(module)}!Epigrass.simobj.graph \textit{(class)}!Epigrass.simobj.graph.getIotaindex \textit{(method)}}

    \vspace{0.5ex}

    \begin{boxedminipage}{\textwidth}

    \raggedright \textbf{getIotaindex}(\textit{self})

    \vspace{-1.5ex}

    \rule{\textwidth}{0.5\fboxrule}
    Returns the Iota index of the graph

    Measures the ratio between the network and its weighed vertices. It 
    considers the structure, the length and the function of a graph and it 
    is mainly used when data about traffic is not available.

    It divides the length of a graph (L(G)) by its weight (W(G)). The lower
    its value, the more efficient the network is. This measure is based on 
    the fact that an intersection (represented as a node) of a high order 
    is able to handle large amounts of traffic.

    The weight of all nodes in the graph (W(G)) is the summation of each 
    node's order (o) multiplied by 2 for all orders above 1.

    \vspace{1ex}

    \end{boxedminipage}

    \label{Epigrass:simobj:graph:getWeight}
    \index{Epigrass \textit{(package)}!Epigrass.simobj \textit{(module)}!Epigrass.simobj.graph \textit{(class)}!Epigrass.simobj.graph.getWeight \textit{(method)}}

    \vspace{0.5ex}

    \begin{boxedminipage}{\textwidth}

    \raggedright \textbf{getWeight}(\textit{self})

    \vspace{-1.5ex}

    \rule{\textwidth}{0.5\fboxrule}
    The weight of all nodes in the graph (W(G)) is the summation of each 
    node's order (o) multiplied by 2 for all orders above 1.

    \vspace{1ex}

    \end{boxedminipage}

    \label{Epigrass:simobj:graph:getLength}
    \index{Epigrass \textit{(package)}!Epigrass.simobj \textit{(module)}!Epigrass.simobj.graph \textit{(class)}!Epigrass.simobj.graph.getLength \textit{(method)}}

    \vspace{0.5ex}

    \begin{boxedminipage}{\textwidth}

    \raggedright \textbf{getLength}(\textit{self})

    \vspace{-1.5ex}

    \rule{\textwidth}{0.5\fboxrule}
    Sum of the length in kilometers of all edges in the graph.

    \vspace{1ex}

    \end{boxedminipage}

    \label{Epigrass:simobj:graph:getPiIndex}
    \index{Epigrass \textit{(package)}!Epigrass.simobj \textit{(module)}!Epigrass.simobj.graph \textit{(class)}!Epigrass.simobj.graph.getPiIndex \textit{(method)}}

    \vspace{0.5ex}

    \begin{boxedminipage}{\textwidth}

    \raggedright \textbf{getPiIndex}(\textit{self})

    \vspace{-1.5ex}

    \rule{\textwidth}{0.5\fboxrule}
    Returns the Pi index of the graph.

    The relationship between the total length of the graph L(G) and the 
    distance along the diameter D(d).

    It is labeled as Pi because of its similarity with the real Pi (3.14), 
    which is expressing the ratio between the circumference and the 
    diameter of a circle.

    A high index shows a developed network. It is a measure of distance per
    units of diameter and an indicator of the  shape of a network.

    \vspace{1ex}

    \end{boxedminipage}

    \label{Epigrass:simobj:graph:getBetaIndex}
    \index{Epigrass \textit{(package)}!Epigrass.simobj \textit{(module)}!Epigrass.simobj.graph \textit{(class)}!Epigrass.simobj.graph.getBetaIndex \textit{(method)}}

    \vspace{0.5ex}

    \begin{boxedminipage}{\textwidth}

    \raggedright \textbf{getBetaIndex}(\textit{self})

    \vspace{-1.5ex}

    \rule{\textwidth}{0.5\fboxrule}
    The Beta index measures the level of connectivity in a graph and is 
    expressed by the relationship between the number of links (e) over the 
    number of nodes (v).

    Trees and simple networks have Beta value of less than one. A connected
    network with one cycle has a value of 1. More complex networks have a 
    value greater than 1. In a network with a fixed number of nodes, the 
    higher the number of links, the higher the number of paths possible in 
    the network. Complex networks have a high value of Beta.

    \vspace{1ex}

    \end{boxedminipage}

    \label{Epigrass:simobj:graph:getAlphaIndex}
    \index{Epigrass \textit{(package)}!Epigrass.simobj \textit{(module)}!Epigrass.simobj.graph \textit{(class)}!Epigrass.simobj.graph.getAlphaIndex \textit{(method)}}

    \vspace{0.5ex}

    \begin{boxedminipage}{\textwidth}

    \raggedright \textbf{getAlphaIndex}(\textit{self})

    \vspace{-1.5ex}

    \rule{\textwidth}{0.5\fboxrule}
    The Alpha index is a measure of connectivity which evaluates the number
    of cycles in a graph in comparison with the maximum number of cycles. 
    The higher the alpha index, the more a network is connected. Trees and 
    simple networks will have a value of 0. A value of 1 indicates a 
    completely connected network.

    Measures the level of connectivity independently of the number of 
    nodes. It is very rare that a network will have an alpha value of 1, 
    because this would imply very serious redundancies.

    \vspace{1ex}

    \end{boxedminipage}

    \label{Epigrass:simobj:graph:getGammaIndex}
    \index{Epigrass \textit{(package)}!Epigrass.simobj \textit{(module)}!Epigrass.simobj.graph \textit{(class)}!Epigrass.simobj.graph.getGammaIndex \textit{(method)}}

    \vspace{0.5ex}

    \begin{boxedminipage}{\textwidth}

    \raggedright \textbf{getGammaIndex}(\textit{self})

    \vspace{-1.5ex}

    \rule{\textwidth}{0.5\fboxrule}
    The Gamma index is a A measure of connectivity that considers the 
    relationship between the number of observed links and the number of 
    possible links.

    The value of gamma is between 0 and 1 where a value of 1 indicates a 
    completely connected network and would be extremely unlikely in 
    reality. Gamma is an efficient value to measure the progression of a 
    network in time.

    \vspace{1ex}

    \end{boxedminipage}

    \label{Epigrass:simobj:graph:doStats}
    \index{Epigrass \textit{(package)}!Epigrass.simobj \textit{(module)}!Epigrass.simobj.graph \textit{(class)}!Epigrass.simobj.graph.doStats \textit{(method)}}

    \vspace{0.5ex}

    \begin{boxedminipage}{\textwidth}

    \raggedright \textbf{doStats}(\textit{self})

    \vspace{-1.5ex}

    \rule{\textwidth}{0.5\fboxrule}
    Generate the descriptive stats about the graph.

    \vspace{1ex}

    \end{boxedminipage}

    \label{Epigrass:simobj:graph:plotDegreeDist}
    \index{Epigrass \textit{(package)}!Epigrass.simobj \textit{(module)}!Epigrass.simobj.graph \textit{(class)}!Epigrass.simobj.graph.plotDegreeDist \textit{(method)}}

    \vspace{0.5ex}

    \begin{boxedminipage}{\textwidth}

    \raggedright \textbf{plotDegreeDist}(\textit{self}, \textit{cum}=\texttt{False})

    \vspace{-1.5ex}

    \rule{\textwidth}{0.5\fboxrule}
    Plots the Degree distribution of the graph maybe cumulative or not.

    \vspace{1ex}

    \end{boxedminipage}

    \label{Epigrass:simobj:graph:getMedianSurvival}
    \index{Epigrass \textit{(package)}!Epigrass.simobj \textit{(module)}!Epigrass.simobj.graph \textit{(class)}!Epigrass.simobj.graph.getMedianSurvival \textit{(method)}}

    \vspace{0.5ex}

    \begin{boxedminipage}{\textwidth}

    \raggedright \textbf{getMedianSurvival}(\textit{self})

    \vspace{-1.5ex}

    \rule{\textwidth}{0.5\fboxrule}
    Returns the time taken by the epidemic to reach 50\% of the nodes.

    \vspace{1ex}

    \end{boxedminipage}

    \label{Epigrass:simobj:graph:getTotVaccinated}
    \index{Epigrass \textit{(package)}!Epigrass.simobj \textit{(module)}!Epigrass.simobj.graph \textit{(class)}!Epigrass.simobj.graph.getTotVaccinated \textit{(method)}}

    \vspace{0.5ex}

    \begin{boxedminipage}{\textwidth}

    \raggedright \textbf{getTotVaccinated}(\textit{self})

    \vspace{-1.5ex}

    \rule{\textwidth}{0.5\fboxrule}
    Returns the total number of vaccinated.

    \vspace{1ex}

    \end{boxedminipage}

    \label{Epigrass:simobj:graph:getTotQuarantined}
    \index{Epigrass \textit{(package)}!Epigrass.simobj \textit{(module)}!Epigrass.simobj.graph \textit{(class)}!Epigrass.simobj.graph.getTotQuarantined \textit{(method)}}

    \vspace{0.5ex}

    \begin{boxedminipage}{\textwidth}

    \raggedright \textbf{getTotQuarantined}(\textit{self})

    \vspace{-1.5ex}

    \rule{\textwidth}{0.5\fboxrule}
    Returns the total number of quarantined individuals.

    \vspace{1ex}

    \end{boxedminipage}

    \label{Epigrass:simobj:graph:getEpistats}
    \index{Epigrass \textit{(package)}!Epigrass.simobj \textit{(module)}!Epigrass.simobj.graph \textit{(class)}!Epigrass.simobj.graph.getEpistats \textit{(method)}}

    \vspace{0.5ex}

    \begin{boxedminipage}{\textwidth}

    \raggedright \textbf{getEpistats}(\textit{self})

    \vspace{-1.5ex}

    \rule{\textwidth}{0.5\fboxrule}
    Returns a list of all epidemiologically related stats.

    \vspace{1ex}

    \end{boxedminipage}

    \label{Epigrass:simobj:graph:getInfectedCities}
    \index{Epigrass \textit{(package)}!Epigrass.simobj \textit{(module)}!Epigrass.simobj.graph \textit{(class)}!Epigrass.simobj.graph.getInfectedCities \textit{(method)}}

    \vspace{0.5ex}

    \begin{boxedminipage}{\textwidth}

    \raggedright \textbf{getInfectedCities}(\textit{self})

    \vspace{-1.5ex}

    \rule{\textwidth}{0.5\fboxrule}
    Returns the number of infected cities.

    \vspace{1ex}

    \end{boxedminipage}

    \label{Epigrass:simobj:graph:getEpisize}
    \index{Epigrass \textit{(package)}!Epigrass.simobj \textit{(module)}!Epigrass.simobj.graph \textit{(class)}!Epigrass.simobj.graph.getEpisize \textit{(method)}}

    \vspace{0.5ex}

    \begin{boxedminipage}{\textwidth}

    \raggedright \textbf{getEpisize}(\textit{self})

    \vspace{-1.5ex}

    \rule{\textwidth}{0.5\fboxrule}
    Returns the size of the epidemic

    \vspace{1ex}

    \end{boxedminipage}

    \label{Epigrass:simobj:graph:getEpispeed}
    \index{Epigrass \textit{(package)}!Epigrass.simobj \textit{(module)}!Epigrass.simobj.graph \textit{(class)}!Epigrass.simobj.graph.getEpispeed \textit{(method)}}

    \vspace{0.5ex}

    \begin{boxedminipage}{\textwidth}

    \raggedright \textbf{getEpispeed}(\textit{self})

    \vspace{-1.5ex}

    \rule{\textwidth}{0.5\fboxrule}
    Returns the epidemic spreading speed.

    \vspace{1ex}

    \end{boxedminipage}

    \label{Epigrass:simobj:graph:getSpreadTime}
    \index{Epigrass \textit{(package)}!Epigrass.simobj \textit{(module)}!Epigrass.simobj.graph \textit{(class)}!Epigrass.simobj.graph.getSpreadTime \textit{(method)}}

    \vspace{0.5ex}

    \begin{boxedminipage}{\textwidth}

    \raggedright \textbf{getSpreadTime}(\textit{self})

    \vspace{-1.5ex}

    \rule{\textwidth}{0.5\fboxrule}
    Returns the duration of the epidemic in units of time.

    \vspace{1ex}

    \end{boxedminipage}

    \label{Epigrass:simobj:graph:resetStats}
    \index{Epigrass \textit{(package)}!Epigrass.simobj \textit{(module)}!Epigrass.simobj.graph \textit{(class)}!Epigrass.simobj.graph.resetStats \textit{(method)}}

    \vspace{0.5ex}

    \begin{boxedminipage}{\textwidth}

    \raggedright \textbf{resetStats}(\textit{self})

    \vspace{-1.5ex}

    \rule{\textwidth}{0.5\fboxrule}
    Resets all graph related stats

    \vspace{1ex}

    \end{boxedminipage}

    \index{Epigrass \textit{(package)}!Epigrass.simobj \textit{(module)}!Epigrass.simobj.graph \textit{(class)}|)}

%%%%%%%%%%%%%%%%%%%%%%%%%%%%%%%%%%%%%%%%%%%%%%%%%%%%%%%%%%%%%%%%%%%%%%%%%%%
%%                           Class Description                           %%
%%%%%%%%%%%%%%%%%%%%%%%%%%%%%%%%%%%%%%%%%%%%%%%%%%%%%%%%%%%%%%%%%%%%%%%%%%%

    \index{Epigrass \textit{(package)}!Epigrass.simobj \textit{(module)}!Epigrass.simobj.line \textit{(class)}|(}
\subsection{Class line}

    \label{Epigrass:simobj:line}
Basic line object containing attributes and methods common to all line 
objects.


%%%%%%%%%%%%%%%%%%%%%%%%%%%%%%%%%%%%%%%%%%%%%%%%%%%%%%%%%%%%%%%%%%%%%%%%%%%
%%                                Methods                                %%
%%%%%%%%%%%%%%%%%%%%%%%%%%%%%%%%%%%%%%%%%%%%%%%%%%%%%%%%%%%%%%%%%%%%%%%%%%%

  \subsubsection{Methods}

    \label{Epigrass:simobj:line:__init__}
    \index{Epigrass \textit{(package)}!Epigrass.simobj \textit{(module)}!Epigrass.simobj.line \textit{(class)}!Epigrass.simobj.line.\_\_init\_\_ \textit{(method)}}

    \vspace{0.5ex}

    \begin{boxedminipage}{\textwidth}

    \raggedright \textbf{\_\_init\_\_}(\textit{self}, \textit{nodes})

    \vspace{-1.5ex}

    \rule{\textwidth}{0.5\fboxrule}
    define a line based on sequence of nodes.

    \vspace{1ex}

    \end{boxedminipage}

    \label{Epigrass:simobj:line:intersect}
    \index{Epigrass \textit{(package)}!Epigrass.simobj \textit{(module)}!Epigrass.simobj.line \textit{(class)}!Epigrass.simobj.line.intersect \textit{(method)}}

    \vspace{0.5ex}

    \begin{boxedminipage}{\textwidth}

    \raggedright \textbf{intersect}(\textit{self})

    \vspace{-1.5ex}

    \rule{\textwidth}{0.5\fboxrule}
    finds out if this line intersects another

    \vspace{1ex}

    \end{boxedminipage}

    \label{Epigrass:simobj:line:xarea}
    \index{Epigrass \textit{(package)}!Epigrass.simobj \textit{(module)}!Epigrass.simobj.line \textit{(class)}!Epigrass.simobj.line.xarea \textit{(method)}}

    \vspace{0.5ex}

    \begin{boxedminipage}{\textwidth}

    \raggedright \textbf{xarea}(\textit{self})

    \vspace{-1.5ex}

    \rule{\textwidth}{0.5\fboxrule}
    finds out if this line extends to more than one area.

    \vspace{1ex}

    \end{boxedminipage}

    \index{Epigrass \textit{(package)}!Epigrass.simobj \textit{(module)}!Epigrass.simobj.line \textit{(class)}|)}

%%%%%%%%%%%%%%%%%%%%%%%%%%%%%%%%%%%%%%%%%%%%%%%%%%%%%%%%%%%%%%%%%%%%%%%%%%%
%%                           Class Description                           %%
%%%%%%%%%%%%%%%%%%%%%%%%%%%%%%%%%%%%%%%%%%%%%%%%%%%%%%%%%%%%%%%%%%%%%%%%%%%

    \index{Epigrass \textit{(package)}!Epigrass.simobj \textit{(module)}!Epigrass.simobj.area \textit{(class)}|(}
\subsection{Class area}

    \label{Epigrass:simobj:area}
Basic area object containing attributes and methods common to all area 
objects.


%%%%%%%%%%%%%%%%%%%%%%%%%%%%%%%%%%%%%%%%%%%%%%%%%%%%%%%%%%%%%%%%%%%%%%%%%%%
%%                                Methods                                %%
%%%%%%%%%%%%%%%%%%%%%%%%%%%%%%%%%%%%%%%%%%%%%%%%%%%%%%%%%%%%%%%%%%%%%%%%%%%

  \subsubsection{Methods}

    \label{Epigrass:simobj:area:__init__}
    \index{Epigrass \textit{(package)}!Epigrass.simobj \textit{(module)}!Epigrass.simobj.area \textit{(class)}!Epigrass.simobj.area.\_\_init\_\_ \textit{(method)}}

    \vspace{0.5ex}

    \begin{boxedminipage}{\textwidth}

    \raggedright \textbf{\_\_init\_\_}(\textit{self}, \textit{nodes})

    \end{boxedminipage}

    \label{Epigrass:simobj:area:centroid}
    \index{Epigrass \textit{(package)}!Epigrass.simobj \textit{(module)}!Epigrass.simobj.area \textit{(class)}!Epigrass.simobj.area.centroid \textit{(method)}}

    \vspace{0.5ex}

    \begin{boxedminipage}{\textwidth}

    \raggedright \textbf{centroid}(\textit{self})

    \vspace{-1.5ex}

    \rule{\textwidth}{0.5\fboxrule}
    calculates the centroid of the area.

    \vspace{1ex}

    \end{boxedminipage}

    \label{Epigrass:simobj:area:area}
    \index{Epigrass \textit{(package)}!Epigrass.simobj \textit{(module)}!Epigrass.simobj.area \textit{(class)}!Epigrass.simobj.area.area \textit{(method)}}

    \vspace{0.5ex}

    \begin{boxedminipage}{\textwidth}

    \raggedright \textbf{area}(\textit{self})

    \vspace{-1.5ex}

    \rule{\textwidth}{0.5\fboxrule}
    calculates the area

    \vspace{1ex}

    \end{boxedminipage}

    \label{Epigrass:simobj:area:neighbors}
    \index{Epigrass \textit{(package)}!Epigrass.simobj \textit{(module)}!Epigrass.simobj.area \textit{(class)}!Epigrass.simobj.area.neighbors \textit{(method)}}

    \vspace{0.5ex}

    \begin{boxedminipage}{\textwidth}

    \raggedright \textbf{neighbors}(\textit{self})

    \vspace{-1.5ex}

    \rule{\textwidth}{0.5\fboxrule}
    return neighbors

    \vspace{1ex}

    \end{boxedminipage}

    \label{Epigrass:simobj:area:isIsland}
    \index{Epigrass \textit{(package)}!Epigrass.simobj \textit{(module)}!Epigrass.simobj.area \textit{(class)}!Epigrass.simobj.area.isIsland \textit{(method)}}

    \vspace{0.5ex}

    \begin{boxedminipage}{\textwidth}

    \raggedright \textbf{isIsland}(\textit{self})

    \vspace{-1.5ex}

    \rule{\textwidth}{0.5\fboxrule}
    return true if the area is an island, i.e. has only one nejghboor and 
    is not on a border.

    \vspace{1ex}

    \end{boxedminipage}

    \label{Epigrass:simobj:area:borderSizeWith}
    \index{Epigrass \textit{(package)}!Epigrass.simobj \textit{(module)}!Epigrass.simobj.area \textit{(class)}!Epigrass.simobj.area.borderSizeWith \textit{(method)}}

    \vspace{0.5ex}

    \begin{boxedminipage}{\textwidth}

    \raggedright \textbf{borderSizeWith}(\textit{self}, \textit{area})

    \vspace{-1.5ex}

    \rule{\textwidth}{0.5\fboxrule}
    returns the size of the border with a given area.

    \vspace{1ex}

    \end{boxedminipage}

    \label{Epigrass:simobj:area:perimeter}
    \index{Epigrass \textit{(package)}!Epigrass.simobj \textit{(module)}!Epigrass.simobj.area \textit{(class)}!Epigrass.simobj.area.perimeter \textit{(method)}}

    \vspace{0.5ex}

    \begin{boxedminipage}{\textwidth}

    \raggedright \textbf{perimeter}(\textit{self})

    \vspace{-1.5ex}

    \rule{\textwidth}{0.5\fboxrule}
    returns the perimeter of the area

    \vspace{1ex}

    \end{boxedminipage}

    \index{Epigrass \textit{(package)}!Epigrass.simobj \textit{(module)}!Epigrass.simobj.area \textit{(class)}|)}

%%%%%%%%%%%%%%%%%%%%%%%%%%%%%%%%%%%%%%%%%%%%%%%%%%%%%%%%%%%%%%%%%%%%%%%%%%%
%%                           Class Description                           %%
%%%%%%%%%%%%%%%%%%%%%%%%%%%%%%%%%%%%%%%%%%%%%%%%%%%%%%%%%%%%%%%%%%%%%%%%%%%

    \index{Epigrass \textit{(package)}!Epigrass.simobj \textit{(module)}!Epigrass.simobj.priorityDictionary \textit{(class)}|(}
\subsection{Class priorityDictionary}

    \label{Epigrass:simobj:priorityDictionary}
\begin{tabular}{cccccccc}
% Line for object, linespec=[False, False]
\multicolumn{2}{r}{\settowidth{\BCL}{object}\multirow{2}{\BCL}{object}}
&&
&&
  \\\cline{3-3}
  &&\multicolumn{1}{c|}{}
&&
&&
  \\
% Line for dict, linespec=[False]
\multicolumn{4}{r}{\settowidth{\BCL}{dict}\multirow{2}{\BCL}{dict}}
&&
  \\\cline{5-5}
  &&&&\multicolumn{1}{c|}{}
&&
  \\
&&&&\multicolumn{2}{l}{\textbf{Epigrass.simobj.priorityDictionary}}
\end{tabular}


%%%%%%%%%%%%%%%%%%%%%%%%%%%%%%%%%%%%%%%%%%%%%%%%%%%%%%%%%%%%%%%%%%%%%%%%%%%
%%                                Methods                                %%
%%%%%%%%%%%%%%%%%%%%%%%%%%%%%%%%%%%%%%%%%%%%%%%%%%%%%%%%%%%%%%%%%%%%%%%%%%%

  \subsubsection{Methods}

    \vspace{0.5ex}

    \begin{boxedminipage}{\textwidth}

    \raggedright \textbf{\_\_init\_\_}(\textit{self})

    \vspace{-1.5ex}

    \rule{\textwidth}{0.5\fboxrule}
    by David Eppstein. 
    {\textless}http://aspn.activestate.com/ASPN/Cookbook/Python/Recipe/117228{\textgreater}
    Initialize priorityDictionary by creating binary heap of pairs 
    (value,key).  Note that changing or removing a dict entry will not 
    remove the old pair from the heap until it is found by smallest() or 
    until the heap is rebuilt.

    \vspace{1ex}

      \textbf{Return Value}
      \begin{quote}
\begin{alltt}
new empty dictionary
\end{alltt}

      \end{quote}

    \vspace{1ex}

      Overrides: dict.\_\_init\_\_

    \end{boxedminipage}

    \label{Epigrass:simobj:priorityDictionary:smallest}
    \index{Epigrass \textit{(package)}!Epigrass.simobj \textit{(module)}!Epigrass.simobj.priorityDictionary \textit{(class)}!Epigrass.simobj.priorityDictionary.smallest \textit{(method)}}

    \vspace{0.5ex}

    \begin{boxedminipage}{\textwidth}

    \raggedright \textbf{smallest}(\textit{self})

    \vspace{-1.5ex}

    \rule{\textwidth}{0.5\fboxrule}
    Find smallest item after removing deleted items from heap.

    \vspace{1ex}

    \end{boxedminipage}

    \vspace{0.5ex}

    \begin{boxedminipage}{\textwidth}

    \raggedright \textbf{\_\_iter\_\_}(\textit{self})

    \vspace{-1.5ex}

    \rule{\textwidth}{0.5\fboxrule}
    Create destructive sorted iterator of priorityDictionary.

    \vspace{1ex}

      Overrides: dict.\_\_iter\_\_

    \end{boxedminipage}

    \vspace{0.5ex}

    \begin{boxedminipage}{\textwidth}

    \raggedright \textbf{\_\_setitem\_\_}(\textit{self}, \textit{key}, \textit{val})

    \vspace{-1.5ex}

    \rule{\textwidth}{0.5\fboxrule}
    Change value stored in dictionary and add corresponding pair to heap.  
    Rebuilds the heap if the number of deleted items grows too large, to 
    avoid memory leakage.

    \vspace{1ex}

      Overrides: dict.\_\_setitem\_\_

    \end{boxedminipage}

    \vspace{0.5ex}

    \begin{boxedminipage}{\textwidth}

    \raggedright \textbf{update}(\textit{self}, \textit{other})

    Update D from E and F: for k in E: D[k] = E[k] (if E has keys else: for
    (k, v) in E: D[k] = v) then: for k in F: D[k] = F[k]

    \vspace{1ex}

      \textbf{Return Value}
      \begin{quote}
\begin{alltt}
None
\end{alltt}

      \end{quote}

    \vspace{1ex}

      Overrides: dict.update 	extit{(inherited documentation)}

    \end{boxedminipage}

    \vspace{0.5ex}

    \begin{boxedminipage}{\textwidth}

    \raggedright \textbf{setdefault}(\textit{self}, \textit{key}, \textit{val})

    \vspace{-1.5ex}

    \rule{\textwidth}{0.5\fboxrule}
    Reimplement setdefault to call our customized \_\_setitem\_\_.

    \vspace{1ex}

      \textbf{Return Value}
      \begin{quote}
\begin{alltt}
D.get(k,d), also set D[k]=d if k not in D
\end{alltt}

      \end{quote}

    \vspace{1ex}

      Overrides: dict.setdefault

    \end{boxedminipage}

    \label{dict:__cmp__}
    \index{dict.\_\_cmp\_\_ \textit{(function)}}

    \vspace{0.5ex}

    \begin{boxedminipage}{\textwidth}

    \raggedright \textbf{\_\_cmp\_\_}(\textit{x}, \textit{y})

    \vspace{-1.5ex}

    \rule{\textwidth}{0.5\fboxrule}
    cmp(x,y)

    \vspace{1ex}

    \end{boxedminipage}

    \label{dict:__contains__}
    \index{dict.\_\_contains\_\_ \textit{(function)}}

    \vspace{0.5ex}

    \begin{boxedminipage}{\textwidth}

    \raggedright \textbf{\_\_contains\_\_}(\textit{D}, \textit{k})

      \textbf{Return Value}
      \begin{quote}
\begin{alltt}
True if D has a key k, else False
\end{alltt}

      \end{quote}

    \vspace{1ex}

    \end{boxedminipage}

    \label{object:__delattr__}
    \index{object.\_\_delattr\_\_ \textit{(function)}}

    \vspace{0.5ex}

    \begin{boxedminipage}{\textwidth}

    \raggedright \textbf{\_\_delattr\_\_}(\textit{...})

    \vspace{-1.5ex}

    \rule{\textwidth}{0.5\fboxrule}
    x.\_\_delattr\_\_('name') {\textless}=={\textgreater} del x.name

    \vspace{1ex}

    \end{boxedminipage}

    \label{dict:__delitem__}
    \index{dict.\_\_delitem\_\_ \textit{(function)}}

    \vspace{0.5ex}

    \begin{boxedminipage}{\textwidth}

    \raggedright \textbf{\_\_delitem\_\_}(\textit{x}, \textit{y})

    \vspace{-1.5ex}

    \rule{\textwidth}{0.5\fboxrule}
    del x[y]

    \vspace{1ex}

    \end{boxedminipage}

    \label{dict:__eq__}
    \index{dict.\_\_eq\_\_ \textit{(function)}}

    \vspace{0.5ex}

    \begin{boxedminipage}{\textwidth}

    \raggedright \textbf{\_\_eq\_\_}(\textit{x}, \textit{y})

    \vspace{-1.5ex}

    \rule{\textwidth}{0.5\fboxrule}
    x==y

    \vspace{1ex}

    \end{boxedminipage}

    \label{dict:__ge__}
    \index{dict.\_\_ge\_\_ \textit{(function)}}

    \vspace{0.5ex}

    \begin{boxedminipage}{\textwidth}

    \raggedright \textbf{\_\_ge\_\_}(\textit{x}, \textit{y})

    \vspace{-1.5ex}

    \rule{\textwidth}{0.5\fboxrule}
    x{\textgreater}=y

    \vspace{1ex}

    \end{boxedminipage}

    \vspace{0.5ex}

    \begin{boxedminipage}{\textwidth}

    \raggedright \textbf{\_\_getattribute\_\_}(\textit{...})

    \vspace{-1.5ex}

    \rule{\textwidth}{0.5\fboxrule}
    x.\_\_getattribute\_\_('name') {\textless}=={\textgreater} x.name

    \vspace{1ex}

      Overrides: object.\_\_getattribute\_\_

    \end{boxedminipage}

    \label{dict:__getitem__}
    \index{dict.\_\_getitem\_\_ \textit{(function)}}

    \vspace{0.5ex}

    \begin{boxedminipage}{\textwidth}

    \raggedright \textbf{\_\_getitem\_\_}(\textit{x}, \textit{y})

    \vspace{-1.5ex}

    \rule{\textwidth}{0.5\fboxrule}
    x[y]

    \vspace{1ex}

    \end{boxedminipage}

    \label{dict:__gt__}
    \index{dict.\_\_gt\_\_ \textit{(function)}}

    \vspace{0.5ex}

    \begin{boxedminipage}{\textwidth}

    \raggedright \textbf{\_\_gt\_\_}(\textit{x}, \textit{y})

    \vspace{-1.5ex}

    \rule{\textwidth}{0.5\fboxrule}
    x{\textgreater}y

    \vspace{1ex}

    \end{boxedminipage}

    \vspace{0.5ex}

    \begin{boxedminipage}{\textwidth}

    \raggedright \textbf{\_\_hash\_\_}(\textit{x})

    \vspace{-1.5ex}

    \rule{\textwidth}{0.5\fboxrule}
    hash(x)

    \vspace{1ex}

      Overrides: object.\_\_hash\_\_

    \end{boxedminipage}

    \label{dict:__le__}
    \index{dict.\_\_le\_\_ \textit{(function)}}

    \vspace{0.5ex}

    \begin{boxedminipage}{\textwidth}

    \raggedright \textbf{\_\_le\_\_}(\textit{x}, \textit{y})

    \vspace{-1.5ex}

    \rule{\textwidth}{0.5\fboxrule}
    x{\textless}=y

    \vspace{1ex}

    \end{boxedminipage}

    \label{dict:__len__}
    \index{dict.\_\_len\_\_ \textit{(function)}}

    \vspace{0.5ex}

    \begin{boxedminipage}{\textwidth}

    \raggedright \textbf{\_\_len\_\_}(\textit{x})

    \vspace{-1.5ex}

    \rule{\textwidth}{0.5\fboxrule}
    len(x)

    \vspace{1ex}

    \end{boxedminipage}

    \label{dict:__lt__}
    \index{dict.\_\_lt\_\_ \textit{(function)}}

    \vspace{0.5ex}

    \begin{boxedminipage}{\textwidth}

    \raggedright \textbf{\_\_lt\_\_}(\textit{x}, \textit{y})

    \vspace{-1.5ex}

    \rule{\textwidth}{0.5\fboxrule}
    x{\textless}y

    \vspace{1ex}

    \end{boxedminipage}

    \label{dict:__ne__}
    \index{dict.\_\_ne\_\_ \textit{(function)}}

    \vspace{0.5ex}

    \begin{boxedminipage}{\textwidth}

    \raggedright \textbf{\_\_ne\_\_}(\textit{x}, \textit{y})

    \vspace{-1.5ex}

    \rule{\textwidth}{0.5\fboxrule}
    x!=y

    \vspace{1ex}

    \end{boxedminipage}

    \vspace{0.5ex}

    \begin{boxedminipage}{\textwidth}

    \raggedright \textbf{\_\_new\_\_}(\textit{T}, \textit{S}, \textit{...})

      \textbf{Return Value}
      \begin{quote}
\begin{alltt}
a new object with type S, a subtype of T
\end{alltt}

      \end{quote}

    \vspace{1ex}

      Overrides: object.\_\_new\_\_

    \end{boxedminipage}

    \label{object:__reduce__}
    \index{object.\_\_reduce\_\_ \textit{(function)}}

    \vspace{0.5ex}

    \begin{boxedminipage}{\textwidth}

    \raggedright \textbf{\_\_reduce\_\_}(\textit{...})

    \vspace{-1.5ex}

    \rule{\textwidth}{0.5\fboxrule}
    helper for pickle

    \vspace{1ex}

    \end{boxedminipage}

    \label{object:__reduce_ex__}
    \index{object.\_\_reduce\_ex\_\_ \textit{(function)}}

    \vspace{0.5ex}

    \begin{boxedminipage}{\textwidth}

    \raggedright \textbf{\_\_reduce\_ex\_\_}(\textit{...})

    \vspace{-1.5ex}

    \rule{\textwidth}{0.5\fboxrule}
    helper for pickle

    \vspace{1ex}

    \end{boxedminipage}

    \vspace{0.5ex}

    \begin{boxedminipage}{\textwidth}

    \raggedright \textbf{\_\_repr\_\_}(\textit{x})

    \vspace{-1.5ex}

    \rule{\textwidth}{0.5\fboxrule}
    repr(x)

    \vspace{1ex}

      Overrides: object.\_\_repr\_\_

    \end{boxedminipage}

    \label{object:__setattr__}
    \index{object.\_\_setattr\_\_ \textit{(function)}}

    \vspace{0.5ex}

    \begin{boxedminipage}{\textwidth}

    \raggedright \textbf{\_\_setattr\_\_}(\textit{...})

    \vspace{-1.5ex}

    \rule{\textwidth}{0.5\fboxrule}
    x.\_\_setattr\_\_('name', value) {\textless}=={\textgreater} x.name = 
    value

    \vspace{1ex}

    \end{boxedminipage}

    \label{object:__str__}
    \index{object.\_\_str\_\_ \textit{(function)}}

    \vspace{0.5ex}

    \begin{boxedminipage}{\textwidth}

    \raggedright \textbf{\_\_str\_\_}(\textit{x})

    \vspace{-1.5ex}

    \rule{\textwidth}{0.5\fboxrule}
    str(x)

    \vspace{1ex}

    \end{boxedminipage}

    \label{dict:clear}
    \index{dict.clear \textit{(function)}}

    \vspace{0.5ex}

    \begin{boxedminipage}{\textwidth}

    \raggedright \textbf{clear}(\textit{D})

    \vspace{-1.5ex}

    \rule{\textwidth}{0.5\fboxrule}
    Remove all items from D.

    \vspace{1ex}

      \textbf{Return Value}
      \begin{quote}
\begin{alltt}
None
\end{alltt}

      \end{quote}

    \vspace{1ex}

    \end{boxedminipage}

    \label{dict:copy}
    \index{dict.copy \textit{(function)}}

    \vspace{0.5ex}

    \begin{boxedminipage}{\textwidth}

    \raggedright \textbf{copy}(\textit{D})

      \textbf{Return Value}
      \begin{quote}
\begin{alltt}
a shallow copy of D
\end{alltt}

      \end{quote}

    \vspace{1ex}

    \end{boxedminipage}

    \label{dict:fromkeys}
    \index{dict.fromkeys \textit{(function)}}

    \vspace{0.5ex}

    \begin{boxedminipage}{\textwidth}

    \raggedright \textbf{fromkeys}(\textit{dict}, \textit{S}, \textit{v}=\texttt{...})

    \vspace{-1.5ex}

    \rule{\textwidth}{0.5\fboxrule}
    v defaults to None.

    \vspace{1ex}

      \textbf{Return Value}
      \begin{quote}
\begin{alltt}
New dict with keys from S and values equal to v
\end{alltt}

      \end{quote}

    \vspace{1ex}

    \end{boxedminipage}

    \label{dict:get}
    \index{dict.get \textit{(function)}}

    \vspace{0.5ex}

    \begin{boxedminipage}{\textwidth}

    \raggedright \textbf{get}(\textit{D}, \textit{k}, \textit{d}=\texttt{...})

    \vspace{-1.5ex}

    \rule{\textwidth}{0.5\fboxrule}
    d defaults to None.

    \vspace{1ex}

      \textbf{Return Value}
      \begin{quote}
\begin{alltt}
D[k] if k in D, else d
\end{alltt}

      \end{quote}

    \vspace{1ex}

    \end{boxedminipage}

    \label{dict:has_key}
    \index{dict.has\_key \textit{(function)}}

    \vspace{0.5ex}

    \begin{boxedminipage}{\textwidth}

    \raggedright \textbf{has\_key}(\textit{D}, \textit{k})

      \textbf{Return Value}
      \begin{quote}
\begin{alltt}
True if D has a key k, else False
\end{alltt}

      \end{quote}

    \vspace{1ex}

    \end{boxedminipage}

    \label{dict:items}
    \index{dict.items \textit{(function)}}

    \vspace{0.5ex}

    \begin{boxedminipage}{\textwidth}

    \raggedright \textbf{items}(\textit{D})

      \textbf{Return Value}
      \begin{quote}
\begin{alltt}
list of D's (key, value) pairs, as 2-tuples
\end{alltt}

      \end{quote}

    \vspace{1ex}

    \end{boxedminipage}

    \label{dict:iteritems}
    \index{dict.iteritems \textit{(function)}}

    \vspace{0.5ex}

    \begin{boxedminipage}{\textwidth}

    \raggedright \textbf{iteritems}(\textit{D})

      \textbf{Return Value}
      \begin{quote}
\begin{alltt}
an iterator over the (key, value) items of D
\end{alltt}

      \end{quote}

    \vspace{1ex}

    \end{boxedminipage}

    \label{dict:iterkeys}
    \index{dict.iterkeys \textit{(function)}}

    \vspace{0.5ex}

    \begin{boxedminipage}{\textwidth}

    \raggedright \textbf{iterkeys}(\textit{D})

      \textbf{Return Value}
      \begin{quote}
\begin{alltt}
an iterator over the keys of D
\end{alltt}

      \end{quote}

    \vspace{1ex}

    \end{boxedminipage}

    \label{dict:itervalues}
    \index{dict.itervalues \textit{(function)}}

    \vspace{0.5ex}

    \begin{boxedminipage}{\textwidth}

    \raggedright \textbf{itervalues}(\textit{D})

      \textbf{Return Value}
      \begin{quote}
\begin{alltt}
an iterator over the values of D
\end{alltt}

      \end{quote}

    \vspace{1ex}

    \end{boxedminipage}

    \label{dict:keys}
    \index{dict.keys \textit{(function)}}

    \vspace{0.5ex}

    \begin{boxedminipage}{\textwidth}

    \raggedright \textbf{keys}(\textit{D})

      \textbf{Return Value}
      \begin{quote}
\begin{alltt}
list of D's keys
\end{alltt}

      \end{quote}

    \vspace{1ex}

    \end{boxedminipage}

    \label{dict:pop}
    \index{dict.pop \textit{(function)}}

    \vspace{0.5ex}

    \begin{boxedminipage}{\textwidth}

    \raggedright \textbf{pop}(\textit{D}, \textit{k}, \textit{d}=\texttt{...})

    \vspace{-1.5ex}

    \rule{\textwidth}{0.5\fboxrule}
    If key is not found, d is returned if given, otherwise KeyError is 
    raised

    \vspace{1ex}

      \textbf{Return Value}
      \begin{quote}
\begin{alltt}
v, remove specified key and return the corresponding value
\end{alltt}

      \end{quote}

    \vspace{1ex}

    \end{boxedminipage}

    \label{dict:popitem}
    \index{dict.popitem \textit{(function)}}

    \vspace{0.5ex}

    \begin{boxedminipage}{\textwidth}

    \raggedright \textbf{popitem}(\textit{D})

    \vspace{-1.5ex}

    \rule{\textwidth}{0.5\fboxrule}
    2-tuple; but raise KeyError if D is empty

    \vspace{1ex}

      \textbf{Return Value}
      \begin{quote}
\begin{alltt}
(k, v), remove and return some (key, value) pair as a
\end{alltt}

      \end{quote}

    \vspace{1ex}

    \end{boxedminipage}

    \label{dict:values}
    \index{dict.values \textit{(function)}}

    \vspace{0.5ex}

    \begin{boxedminipage}{\textwidth}

    \raggedright \textbf{values}(\textit{D})

      \textbf{Return Value}
      \begin{quote}
\begin{alltt}
list of D's values
\end{alltt}

      \end{quote}

    \vspace{1ex}

    \end{boxedminipage}


%%%%%%%%%%%%%%%%%%%%%%%%%%%%%%%%%%%%%%%%%%%%%%%%%%%%%%%%%%%%%%%%%%%%%%%%%%%
%%                              Properties                               %%
%%%%%%%%%%%%%%%%%%%%%%%%%%%%%%%%%%%%%%%%%%%%%%%%%%%%%%%%%%%%%%%%%%%%%%%%%%%

  \subsubsection{Properties}

\begin{longtable}{|p{.30\textwidth}|p{.62\textwidth}|l}
\cline{1-2}
\cline{1-2} \centering \textbf{Name} & \centering \textbf{Description}& \\
\cline{1-2}
\endhead\cline{1-2}\multicolumn{3}{r}{\small\textit{continued on next page}}\\\endfoot\cline{1-2}
\endlastfoot\raggedright \_\-\_\-c\-l\-a\-s\-s\-\_\-\_\- & \raggedright \textbf{Value:} 
{\tt {\textless}attribute '\_\_class\_\_' of 'object' objects{\textgreater}}&\\
\cline{1-2}
\end{longtable}

    \index{Epigrass \textit{(package)}!Epigrass.simobj \textit{(module)}!Epigrass.simobj.priorityDictionary \textit{(class)}|)}
    \index{Epigrass \textit{(package)}!Epigrass.simobj \textit{(module)}|)}
