\chapter{Example of Model Definition Script} 
\label{script}

\begin{lstlisting}[basicstyle=\footnotesize,language=Python, frame=trBL, caption= ,label=]

################################################################
#
#  EPIGRASS -Model Definition
#  This script describes model and parameters specified
#  by the user.
#  It can be edited by the user directly, by means of a text editor.
#  WARNING: No variables may be removed, even if not used by the chosen model . 
#  Any comments added by the user must be preceeded  by the symbol #
#
################################################################
################################################################

#==============================================================#
[THE WORLD]
#==============================================================#
# Here you add iformation about the files that described the world the model is enclosed
# "Location" is the GRASS location with the maps describing the area.
# "vector layer" is the map layer that will be used as a background for the network display
# "sites" and "edges" are the files that describe the topology of the network (see Documtntation)

location = BRASIL
vector layer = Limite_politico_administrativo
sites = sitios2.csv
edges = edgesout.csv

#==============================================================#
[EPIDEMIOLOGICAL MODEL]
#==============================================================#
#model types available: SIS, SIS_s ,SIR, SIR_s, SEIS, SEIS_s, SEIR, SEIR_s, 
# SIpRpS, SIpRpS_s,SIpR,SIpR_s(see documentation for description)
modtype = SIR

#==============================================================#
[MODEL PARAMETERS]
#==============================================================#

#  They can be specified as constants or as functions of global or 
#  site-specific variables. these site-specific variables, are provided
#  in the sites file. All the numbers given after the geocode (4th column)
#  are collected into the values tuple.
# Examples:
# beta = 0.001
# beta=values[0] #assigns the first element of values to beta 
# beta=0.001*values[1]

beta = 0.4   #transmission coefficient (contact rate * transmissibility)
alpha = 1  # clumping parameter
e = 1   # inverse of incubation period
r = 0.1   # inverse of infectious period
delta = 1  # probability of acquiring full immunity [0,1]
B = 0           # Birth rate
w = 0           # probability of immunity waning [0,1]
p = 0           # Probability of a recovered become infected per time step [0,1]


#==============================================================#
[INITIAL CONDITIONS]
#==============================================================#
# Here, the number of individuals in each epidemiological
# state (SEI) is specified. They can be specified in absolute
# or relative numbers.
# N is the population size of each site.
# The rule defined here will be applied equally to all sites.
# For site-specific definitions, use EVENTS (below)
# Examples:
# S,E,I = 0.8*N, 10, 0.5*E
# S,E,I = 0.5*N, 0.01*N, 0.05*N
# S,E,I = N-1, 1, 0
S = N
E = 0
I = 0

#=============================================================#
[EPIDEMIC EVENTS]
#=============================================================#
# Specify isolated events.
#  Localities where the events are to take place should be Identified by the geocode, which 
#  comes after population size on the sites data file.
#  All coverages must be a number between 0 and 1.
#  Seed : [('locality1's geocode', n),('locality2's geocode', n)]. 
#  N infected cases will be added to locality at time 0.
#  Vaccinate: [('locality1's geocode', time, coverage),('locality2's geocode', time, coverage)] 
#  Quarantine: [(locality1's geocode,time,coverage), (locality2's geocode,time,coverage)]
seed = [(230440005,1)] #Fortaleza
Vaccinate = [(355030800, 31, 0)] #Sao Paulo
Quarantine = [(355030800,20,0)]
#Screening = (locality, time, coverage)#screening for sick people on aiports bus stations
#Vector_control = (locality, time, coverage)
#Treatment = (locality, time, coverage)
#Prophylaxis = (locality, time, coverage)

#==============================================================#
[TRANSPORTATION MODEL]
#==============================================================#
# If doTransp = 1 the transportation dinamics will be included. Use 0 here only for debugging purposes.
doTransp = 1

# NEIGHBORHOOD DEFINITION


# SPATIAL COUPLING


#==============================================================#
[SIMULATION AND OUTPUT]
#==============================================================#

# Number of steps
steps = 200

# Output dir
outdir = ~/data

# Output file
outfile = simul.dat

# Database Output
# MySQLout can be 0 (no database output) or 1
MySQLout = 1 

# Graphical outputs
draw map = 1

# Report Generation
# The variable report can take the following values:
# 0 - No report is generated.
# 1 - A network analysis report is generated in PDF Format.
# 2 - An epidemiological report is generated in PDF Format.
# 3 - A full report is generated in PDF Format.
# siteRep is a list with site geocodes. For each site in this list, a detailed report is apended to the main report.
report = 0
siteRep = [230440005,355030800]

#Batch Run
#  list other scripts to be run in after this one. don't forget the extension .epg
#  model scripts must be in the same directory as this file or provide full path.
#  Example: Batch = ['model2.epg','model3.epg','/home/jose/model4.epg']
Batch = []

################################################################
################################################################



\end{lstlisting}